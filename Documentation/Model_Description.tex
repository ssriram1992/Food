\documentclass[letter,12pt]{article}

\usepackage{graphics}
\usepackage{graphicx} 
\usepackage{rotating}
\usepackage{dcolumn}
\usepackage{longtable}
\usepackage{multirow}
\usepackage{amsmath}
%\usepackage{amsthm}
\usepackage{amssymb}
\usepackage{xspace}
\usepackage{textcase}
\usepackage[margin=1in]{geometry}


\begin{document}
\title{MCP Food Systems Model: Mathematical Model Description}
\author{}
\date{}

\maketitle

The model described in this working paper builds upon and advances the state-of-the-art in several ways.  Firstly, our model captures important food supply chain components: we explicitly model trade and food distribution in a way which accounts for infrastructure and geography.  Most global food models have production feeding into a global trade pool from which food is allocated to various countries and which produces global commodity prices.  Capturing trade and food distribution in this way enables us to account for transportation costs and regional price variations.  Distribution is, in fact, one of the key areas in which food and energy markets interact, and our model is designed to integrate with an existing energy model at precisely this point (among others).

Secondly, our model considers food access and utilization.  Part of this consideration consists of food waste modelling, and another part of it includes disaggregating consumption by income and age; this allows us to provide more detailed and more accurate information regarding nutrition and public health in general.  Thirdly, the model is set up to evaluate the effects of seasonality and system shocks.  Using a monthly time-step and taking storage capacity (and potential food aid) into account will show how food access varies througout the year as well as across years, and the potential buffering effects of storage and food aid promote a more realistic model response to shocks like crop failure.

Finally, we use microeconomic utility maximization to model food commodities (and energy commodities, in the energy model we are using) which results in a Mixed Complementarity Problem (MCP).  This is a fundamentally different modelling approach than most global food models, so its system dynamics may also be different; among other things, it forces the decision modelling to be more explicit while also capturing food switching by the players in the multi-player game.

All of this matters because food security and nutrition are not simply functions of production.  Prices matter, and so does infrastructure: the latter affects the former, and both in turn affect food consumption and consumer switching between food commodities.  Secondly, the energy impacts on food are also significant because both production and distribution use energy; integrating microeconomic food and energy models also positions us nicely to model the effects of biofuels in both contexts.  Thirdly, there are important variations in time, throughout the year, as well in space, and those variations are important for understanding food security.  Finally, food waste and utilization considerations mean that not all of the nutrition in the food produced actually gets used.  Consumption point estimates, moreover, are insufficient in this context -- age and income distributions are necessary to capture human nutrition accurately.

\pagebreak



\section{Energy}

The price of fuel, $p_{fuel}$ is an exogenous input from the global energy model.

\section{Climate}

Our climate model needs to output the following values on a daily basis: $d_{precip}$ (total rain each days), $E_{PAR}$ (average density of photosynthetically active radiation each day), $T_{max}$ (maximum temperature), $T_{min}$ (minimum temperature), $T_{mean}$ (mean temperature), $T_{wb}$ (mean wet bulb temperature), and $T_{db}$ (mean dry bulb temperature)

\section{Water}

Water is primarily an input to other things, so we should just be able to use baseline values from a scenario and measure water withdrawals for irrigation, runoff, etc.


\section{Soil}

Assume that soil can be divided up into 3 classes of soil quality $s_q$ and 3 classes of soil fertility $s_f$, each with high, medium, and low values ($s_q$ or $s_f$ = 1, 2, or 3, respectively).  Soil quality affects water retention (i.e. how much of the precipitation the soil can hold for the crops) and final yield.  The soil quality can then be changed over time through tillage practices (possibly specify this exogenously) and fertilizer use (we will not consider this yet).  

For final yield, 

\begin{equation}
r_{sq} = \left\{ \begin{array}{cc}
1			&s_q = 1 \\
0.7		&s_q = 2 \\
0.4		&s_q = 3
\end{array}
\right.
\end{equation}

Further assume that natural nutrient availability depends on $s_f$:

\begin{align}
\rho_{N} \quad &= \quad  \rho_{N} \left(s_f\right) \\
\rho_{P} \quad &= \quad  \rho_{P} \left(s_f\right) \\
\rho_{K} \quad &= \quad  \rho_{K} \left(s_f\right)
\end{align}

Nutrients can then be added to this baseline using fertilizer.  Alternatively, for a simpler model, just have a soil fertility value as is the case for soil quality:

\begin{equation}
r_{sf} = \left\{ \begin{array}{cc}
1			&s_f = 1 \\
0.7		&s_f = 2 \\
0.4		&s_f = 3
\end{array}
\right.
\end{equation}

For water retention, the amount of water available to the plants on a given day, $d_{water,n}$, is

\begin{equation}
d_{water,n} = r_{sq} d_{precip,n} +  r_{sq}^{\alpha_{soil}} d_{water,n-1}
\end{equation}

\noindent where $d_{precip,n}$ is the precipitation, in metres, on day $n$ and $\alpha_{soil}$ is a calibrated parameter.  This model accounts for the role that soil quality plays in the retention of precipitation and existing water in the ground.  A more sophisticated model would take temperature and evapotranspiration into account, too.

\section{Health}

Let nutrient intake be a vector of calories, protein, and micronutrients $\boldsymbol \nu$.  Non-communicable disease rates (Ben Miller's project) and labour productivity (loss) (Matthew Bee's project) are functions of nutrition:

\begin{align}
d_{non-communicable} \quad &= \quad  d_{non-communicable}\left(\boldsymbol \nu\right) \\
r_{labour} \quad &= \quad  r_{labour} \left( \boldsymbol \nu \right)
\end{align}

Consider using a function of the form

\begin{equation}
r_{labour} = \exp \left(- \boldsymbol \alpha_{labour}^T \boldsymbol \nu_{working} \right) + 1
\end{equation}

\noindent where $\boldsymbol \alpha_{labour}$ is a vector parameter to be determined empirically, and $\boldsymbol \nu_{working}$ is the average nutrient intake for working-age males over all income levels.  Consider taking just a lower income bracket, because they will likely be the ones doing the manual labour.  Let $r_{labour,ave}$ be the average value of $r_{labour}$ over the course of a given year.  Both $r_{labour}$ and $r_{labour,ave}$ are state variables.

This could cause computational problems because of its nonlinearity, though.  Therefore, consider an alternate function

\begin{equation}
r_{labour} = 2 - \boldsymbol \alpha_{labour}^T \boldsymbol \nu_{working}
\end{equation}

\noindent with different values for $\boldsymbol \alpha_{labour}$.

For the non-communicable diseases, the primary nutrients we will consider are vitamin A, iron, iodine, calories, and protein, and the main diseases we will monitor are xerophthalmia (with night blindness), anemia, hypthyroidism, stunting, and wasting (and maybe stress).

\section{Demography}

Ignore migration for the time being, and use existing population projections.  We want $\rho_I \left(I\right)$, the per capita income distribution, $\rho_{a} \left(a\right)$, the population's age distribution, and $\rho_g \left(g\right)$, the population's gender distribution.  Ideally, we would use joint distributions instead -- e.g. $\rho_{a,g,I} = \rho_{a,g|I} \left(a,g|I\right) \rho_I \left(I\right)$ instead of $\rho_{a,g,I} = \rho_{a} \left(a\right) \rho_g \left(g\right) \rho_I \left(I\right)$.  Each node in the model would also have its own distribution.

\section{Calendar}

Start each year in February, rather than January -- this lines up better with the crop cycle.  Let $r_{month}$ be the ratio of each month's length to that of a 30-day month (e.g. $r_{month}$ for February would be 28/30).  We will ignore leap years for the time being.  We also need to track the first day of each month

\section{Crop Production}

\subsection{Optimization}
The optimization problem for each region each year is

\begin{align}
\max U \quad &= \quad  \sum_t R\left(t\right) - C - C_{change} \label{crop utility fcn}\\
\Delta A_{conv} \quad &\leq \quad  \Delta A_{conv,max} \label{land conv con} \\
q_{crop,store,i}^C + \sum_j q_{crop,transp,i \rightarrow j,buy}^C \quad &\leq \quad  Y_{crop}  \label{crop sale con} \\
\sum_{crop,season} A_{crop,season} \quad &\leq \quad  A_{tot} \label{land area con}
\end{align}

\noindent with decision variables

\begin{equation}
A_{crop,season},\Delta A_{conv},q_{crop,store,i}^C,q_{crop,transp,i\rightarrow j,buy}^C
\end{equation}

\noindent and state variables

\begin{equation}
A_{tot}, Y_{crop}
\end{equation}

\noindent where

\begin{align}
R\left(t\right) \quad &= \quad  \sum_{crop} p_{crop,store,i} q_{crop,store,i}^C + \sum_{crop} \sum_j p_{crop,transp,i\rightarrow j,buy} q_{crop,transp,i\rightarrow j,buy}^C \\
C \quad &= \quad  C_{labour} + C_{fuel} + C_{expansion} \\
C_{labour} \quad &= \quad  r_{labour,ave} p_{labour} \sum_{crop,season} A_{crop,season} t_{crop} \\
C_{fuel} \quad &= \quad  p_{fuel} \sum_{crop,season} A_{crop,season} q_{fuel,crop} \\
C_{expansion} \quad &= \quad  p_{conv} \Delta A_{conv} \\
C_{change,n} \quad &= \quad  \frac{1}{2} \sum_{crop,season} p_{change,crop} \left( A_{crop,season,n} - A_{crop,season,n-1} \right)^2 \\
A_{tot,n} \quad &= \quad  A_{tot,n-1} + \Delta A_{conv,n} \label{crop area increment}
\end{align}

The utility function (\ref{crop utility fcn}) is the profit, on a yearly basis, minus $C_{change}$.  $C_{change}$ represents the farmers' preference not to change (i.e. to do the same thing they did last year).  (\ref{land conv con}) sets a limit on how much land can be converted to agricultural use in a given year.  (\ref{crop sale con}) ensures that the quantities sold in a given month are no greater than the harvest for that month.  (\ref{land area con}) ensures that the total area cropped is no greater than the total available area (as defined by (\ref{crop area increment})).  

Assuming perfect foreknowledge, the optimization problem for an area then becomes a linear programming problem plus a quadratic term, $C_{change}$, to limit the change in land use from year to year.

The equilibrium conditions are

\begin{align}
q_{crop,transp,i\rightarrow j,buy}^C - q_{crop,transp,i\rightarrow j,buy}^D \quad &= \quad  0\\
q_{crop,store,i}^C - q_{crop,store,i}^S \quad &= \quad  0
\end{align}

The duals for these equations are $p_{crop,transp,i\rightarrow j,buy}$ and $p_{crop,store,i}$, respectively.

This optimization happens on a yearly basis, but in order to make the rolling time horizon work for livestock producer and storage optimizations, we use the three-year optimization

\begin{equation}
\max U + \delta \tilde{U} + \delta^2 \hat{U}
\end{equation}

\noindent where $\tilde{U}$ is the advisory utility for the second year, and $\hat{U}$ is the advisory utility for the third year.  The constraints and update equations hold on their particular time-step (i.e. monthly or yearly) for all three years.


\subsection{Crops}

If planting times $d_{plant}$ are fixed (or calculated from weather data), yields and harvest times can be calculated from weather data.

Let the sowing date is determined by soil moisture: sowing happens $\Delta d_{early}$ days before soil moisture reaches $d_{water,threshold}$.  Thus

\begin{align}
d_{plant} &= d_{thresh} - \Delta d_{early} \\
d_{water} \left(t = d_{thresh} - 1 \right)& < d_{water,thresh} \leq d_{water} \left(t = d_{thresh} \right)
\end{align}

Each crop, furthermore, has potential harvests in the belg and kremt seasons, but assume that land used in the belg season can't be used in the kremt season, and vice versa.  If the potential sowing date lands too late -- after the end of the sowing season $d_{season,end}$ -- there is no crop for that season.  Assume that the potential sowing season is February-March for the belg season and May-June for the kremt season.

Crop growth consists of two parts: vegetative growth and food production/growth.  Both proceed on a daily time step.  For vegetative growth

\begin{align}
L_n \quad &= \quad  L_{n-1} + \Delta L_n \\
\Delta L \quad &= \quad  r_{water} r_{temp} \rho_{plant} \Delta A_{leaf,max} \Delta N \frac{a}{1+a} \\
r_{temp} \quad &= \quad  1 - 0.0025 \left(0.25 T_{min} + 0.75 T_{max} - 26 \right)^2 \\
a \quad &= \quad  e^{2\alpha_1 \left(N-\alpha_2\right)} \\
N_n \quad &= \quad  N_{n-1} + \Delta N_n \\
\Delta N \quad &= \quad  r_{temp} \Delta N_{max}
\end{align}

\noindent where $L$ is leaf area index, $\rho_{plant}$ is the plant density, $\Delta A_{leaf, max}$ is the maximum leaf area expansion per leaf, $N$ is the leaf number (a measure of plant maturity), and the $\alpha$ parameters are empirical constants.

Once $N \geq N_{mature}$, we enter reproductive growth (i.e. seed/food):

\begin{align}
m_{h,n} \quad &= \quad   m_{h,n-1} + \Delta m_{h,n} \\
\Delta m_h \quad &= \quad   2.1 r_{water} r_{temp} E_{PAR} \left( 1 - e^{-Y_1 L}\right) \\
Y_1 \quad &= \quad   1.5 - 0.768 \left[ \left(\delta_{row}\right)^2 \rho_{plant}\right]^{0.1} \\
\Delta L \quad &= \quad   - \rho_{plant} \Delta I \Delta A_{remove} \rho_{SLA} \\
\Delta I \quad &= \quad   \left\{ \begin{array}{cc}
T_{mean} - T_{base} 		& \text{if} \ T_{base} \leq T_{mean} \leq 25 \\
0												&\text{else}
\end{array}\right. \\
I_n \quad &= \quad   I_{n-1} + \Delta I_n \\
L_n \quad &= \quad   L_{n-1} + \Delta L_n
\end{align}

\noindent where $m_h$ is the total fruit/seed mass, $E_{PAR}$ is the density of photosynthetically active radiation, $\delta_{row}$ is the row spacing, $I$ is the accumulated temperature after the reproductive stages starts, $\rho_{SLA}$ is the specific leaf area, and $T_{base}$ is the base temperature above which reproductive growth occurs.  Consider using the a crop-specific extinction coefficient, $K$ (ritchie98col2) instead of the calculated $Y_1$ value.

Consider eventually using $\rho_{plant}$ as a decision variable for the farmer.  Using it now would make the problem very nonlinear, however, so we will not consider that right now.

The crop is mature once $I \geq I_{tot}$, where $I_{tot}$ is the duration of the reproductive stage in degree days.  The crop harvest is then 

\begin{equation}
m_{crop} = \left\{ \begin{array}{cc}
m_h			& \text{in its harvest month} \\
0				& \text{else} 
\end{array} \right.
\end{equation} 

For water, use

\begin{equation}
r_{water} = 1 - \exp \left(-\frac{d_{water} \ln 2}{\alpha_{water} \rho_{plant}} \right)
\end{equation}

\noindent where $\alpha_{water}$ is a crop-specific parameter describing the water needs of the plant; high values indicate high water needs and vice versa.  A more sophisticated model would take temperature and evapotranspiration into account.

\subsection{Fertilizer, Nutrients, and Final Yield}

For now, we will assume that no fertilizer is used and that final yield can be reduced based on soil quality and fertility/nutrient availability after the rest of the crop model has been calculated.  If we use $r_{sf}$, then

\begin{equation}
Y_{crop} = \sum_{season} m_{crop,season} A_{crop,season} r_{sq} r_{sf}
\end{equation} 

If we use explicit nutrient densities $\rho_x$, then
\begin{align}
r_{x,crop} \quad &= \quad  \min \left(\frac{\rho_x}{x_{max,crop}\rho_{plant}},1 \right) \\
Y_{crop} \quad &= \quad  \sum_{season} m_{crop,season} A_{crop,season} r_{sq} \left(\min_x r_{x,crop} \right)
\end{align} 

This is a Leontief production function (with respect to nutrient inputs) -- the smallest $r_x$ value is the one that limits production.  Note that in both cases, however, the only decision variable involved is $A_{crop,season}$, and every other quantity in the expressions is a parameter.

If we incorporate a $q_{fert,crop,season}$ decision variable, however, matters are much more complicated.  Assume that manure is free fertilizer.  The manure available for fertilizer is

\begin{equation}
q_{manure} = N_{cattle} r_{fert} \eta_{fert}
\end{equation}

\noindent where $r_{fert}$ is the amount of manure produced per cow and $\eta_{fert}$ is the fraction of that manure used as fertilizer.  Assume that fertilizer is not transported between regions.  Imported fertilizer prices will be a function of global energy and raw material prices.  The fertilizer used is a combination of purchased fertilizer and manure, and from that, we can calculate $r_{x,crop,season}$:
\begin{align}
q_{fert,used} = q_{fert} + q_{manure} \\
r_{x,crop,season} = \min \left(\frac{\rho_x + \frac{q_{fert,crop,season,used} \rho_{fert,x}}{A_{crop,season}}}{x_{max,crop}\rho_{plant}},1 \right)
\end{align}

\noindent where $\rho_{fert,x}$ is the nutrient density of the fertilizer.  Sum over fertilizers if multiple fertilizers with different nutrient densities are used.  This allows the fertilizer to be allocated to different crops, but it ignores any temporal variation with regards to the fertilizer (i.e. it doesn't matter when you apply it). 

We may consider $r_x^{\phi}$ instead of $r_x$: the $\phi$ values may end up differing depending on the nutrient -- we need more thought here regarding what they should be.  This results in a Cobb-Douglas production function.  If we assume constant returns to scale, then we use $r_{x,crop}^{\phi}$, with $\phi = 1/3$, instead of $r_{x,crop}$.  Then

\begin{equation}
Y_{crop} = \sum_{season} m_{crop,season} A_{crop,season} r_{sq} \prod_x r_x^{\frac{1}{3}}
\end{equation}

This formulation is computationally problematic, however, because the $r_x$ functions are not smooth; the derivative of $r_x^{\frac{1}{3}}$ at 0 is also infinity.  Therefore, we will use a smooth approximation that will give us similar behaviour:
\begin{align}
r_x^{\frac{1}{3}} \approx \tilde{r}_x  \quad &= \quad  1 - \exp \left[ -\left(\frac{\rho_x + \frac{q_{fert,used} \rho_{fert,x}}{A_{crop}}}{\frac{1}{3} x_{max,crop}\rho_{plant}} \right) \right] \\
Y_{crop}  \quad &= \quad  \sum_{season} m_{crop,season} A_{crop,season} r_{sq} \prod_x \tilde{r}_{x,crop,season}
\end{align}

We could do something similar with irrigation as well.

\section{Livestock Production}

\subsection{Optimization}

The optimization problem maximizes profit over a three-year period but only keeps the first year's results:

\begin{gather}
\max \sum_{t=1}^{12} \left(R - C\right) + \sum_{t=13}^{24} \delta \left(\tilde{R} - \tilde{C}\right) + \sum_{t=25}^{36} \delta^2 \left(\hat{R} - \hat{C}\right) + p_{cattle,ave} N_{cattle,i,final}\\
q_{milk,store,i}^L + \sum_j q_{milk,transp,i \rightarrow j,buy}^L \leq q_{milk,i} \label{milk sold con}\\
N_{cattle,i} \geq 0 \label{herd size con}\\
R = p_{milk,store,i} q_{milk,store,i}^L + p_{beef,store,i} q_{beef,store,i}^L + \nonumber \\
\sum_j \left( p_{milk,transp,i \rightarrow j,buy} q_{milk,transp,i \rightarrow j,buy}^L + p_{cattle,transp,i \rightarrow j,buy} q_{cattle,transp,i \rightarrow j,buy}^L \right. \\
\left. + p_{beef,transp,i \rightarrow j,buy} q_{beef,transp,i \rightarrow j,buy}^L \right) + \left(q_{beef,store,i}^L + \sum_j q_{beef,transp,i \rightarrow j,buy}^L \right) \frac{p_{hide}}{m_{cow} r_{meat}} \\
C = \sum_j p_{cattle,transp,j \rightarrow i,sell} q_{cattle,transp,j \rightarrow i,sell}^L + \nonumber \\
\left(r_{feed} \mu_{feed} p_{feed} + r_{labour} p_{labour} t_{cattle} \right) N_{cattle,i}
\end{gather}

\noindent with decision variables

\begin{align}
q_{milk,store,i}^L,q_{beef,store,i}^L,q_{milk,transp,i \rightarrow j}^L,q_{cattle,transp,i \rightarrow j,buy}^L, \nonumber \\
q_{beef,transp,i \rightarrow j}^L,q_{cattle,transp,j \rightarrow i,sell}^L
\end{align}

\noindent and state variable

\begin{equation}
N_{cattle,i}
\end{equation}

\noindent where $R$ is revenue, $C$ is cost, $\mu_{feed}$ is how much food it takes to feed a cow for a month, $r_{feed}$ is the fraction of the cow's food that has to be purchased, $\delta$ is the discount factor, $m_{cow}$ is the average mass per cow, $r_{meat}$ is the fraction of the cow's mass that can be sold as beef, $p_{cattle,ave}$ is the average $p_{cattle}$ value over the three-year period, and $N_{cattle,final}$ is the value of $N_{cattle,i}$ at the end of the optimization.  Therefore, the livestock producer is maximizing profit plus a measure of the final herd value.  $p_{feed}$ is somehow related to the market price of food at node $i$.  $p_{hide}$ is the price of a cow's hide (sold to make leather, etc.).  $q_{milk,i}$ is the total milk production, which will be calculated in the next section.  The tildes and hats indicate that the quantities are calculated using advisory variables.  

(\ref{milk sold con}) ensures that the milk sold to storage or distribution is no greater than the total milk production in the region.  (\ref{herd size con}) ensures that the herd size does not become negative.

The walking bank hypothesis says that herders sell animals to meet immediate cash needs; the animals are productive assets held for long-term equity growth.  Therefore, herders will sell fewer animals as prices increase, increased fixed costs will lead to higher sales (bellemare06jsr).  Check to see how well the model replicates this behaviour.

Eventually, we should distinguish between bulls and cows for weight and meat purposes, and because there is a significant size difference.

The equilibrium conditions are

\begin{align}
q_{milk,store,i}^L - q_{milk,store,i}^S \quad &= \quad  0\\
q_{beef,store,i}^L - q_{beef,store,i}^S \quad &= \quad  0\\
q_{milk,transp,i \rightarrow j,buy}^L - q_{milk,transp,i \rightarrow j,buy}^D \quad &= \quad  0  \\
q_{cattle,transp,i \rightarrow j,buy}^L - q_{cattle,transp,i \rightarrow j,buy}^D \quad &= \quad  0 \\
q_{beef,transp,i \rightarrow j,buy}^L - q_{beef,transp,i \rightarrow j,buy}^D \quad &= \quad  0  \\
q_{cattle,transp,k \rightarrow i,sell}^L - q_{cattle,transp,k \rightarrow i,sell}^D \quad &= \quad  0
\end{align}

The duals of these equations give $p_{milk,store,i}$, $p_{beef,store,i}$, $p_{milk,transp,i \rightarrow j}$, $p_{cattle,transp,i \rightarrow j,buy}$, $p_{beef,transp,i \rightarrow j}$, and $p_{cattle,transp,k \rightarrow i,sell}$, respectively.


\subsection{Production Functions}

Assume $N_{cattle}$ for month $n+1$ to be 

\begin{align}
N_{cattle,i,n+1} = \left(1 + k - \kappa \right) N_{cattle,i,n} - \frac{q_{beef,store,i,n+1}^L}{r_{meat} m_{cow}} \nonumber \\
- \sum_j \left( q_{cattle,transp,i \rightarrow j,buy,n+1}^L - \frac{q_{beef,transp,i \rightarrow j,buy,n+1}^L}{r_{meat} m_{cow}} \right) + \sum_{j} q_{cattle,transp,j\rightarrow i,sell,n+1}
\end{align}

\noindent where $N_{cattle,i}$ is the total number of cattle, $k$ is the natural cattle growth and $\kappa$ is the loss due to disease/predation.  This assumes a constant, 30-day month, however.  Accounting for irregular months leads to

\begin{align}
N_{cattle,i,n+1} = \left(1 + k - \kappa \right)^{r_{month,n+1}} N_{cattle,i,n} - \frac{q_{beef,store,i,n+1}^L}{r_{meat} m_{cow}} \nonumber \\
- \sum_j \left( q_{cattle,transp,i \rightarrow j,buy,n+1}^L - \frac{q_{beef,transp,i \rightarrow j,buy,n+1}^L}{r_{meat} m_{cow}} \right) + \sum_{j} q_{cattle,transp,j\rightarrow i,sell,n+1}
\end{align}

$k-\kappa$ averages around 6.5\% per year (bellemare06jsr); $k-\kappa$ is therefore about $5 \times 10^{-3}$ on a monthly basis.  Dairy production is

\begin{equation}
q_{milk,i} = \eta_{production} r_{dairy} N_{cattle,i} \mu_{production} r_{month}
\end{equation}

\noindent where $\eta_{production}$ is the efficiency of dairy production, $r_{dairy}$ is the fraction of cattle used for dairy, and $\mu_{production}$ is a baseline expected production of milk per cow (per 30-day month).  For pastoralists, $r_{dairy}$ is around 0.67.  $\eta_{production}$ may be a function of cattle nutrition, water availability, and temperature, but based loosely on the results of bohmanova07jsr, we specify efficiency as a function of weather:

\begin{align}
\eta_{production} = 1- \alpha_{dairy} \sum_t \tau_t \\
\tau = \max \left\{\left[1.8 \left(0.35 T_{db} + 0.65 T_{wb}\right) + 32\right] - 74,0\right\}
\end{align}

\noindent where $\alpha_{dairy} = 5.3 \times 10^{-5}$, $\tau$ is the temperature-humidity index for a given day, $T_{db}$ is the dry bulb temperature, and $T_{wb}$ is the wet bulb temperature (both temperatures are daily averages).  $T_{db}$ is simply normal temperature, and $T_{wb}$ may alternatively be calculated from temperature and humidity.


\section{Distribution}

\subsection{Optimization}

The optimization problem tries to maximize profit on the same three-year rolling time horizon, keeping the first year's results only, as livestock production and storage do:

\begin{gather}
\max P = \sum_{t=1}^{12} \left(R - C\right) + \sum_{t=13}^{24} \delta \left(\tilde{R} - \tilde{C}\right) + \sum_{t=25}^{36} \delta^2 \left(\hat{R} - \hat{C}\right) \\
R = \sum_{food} p_{food,transp,i\rightarrow j,sell} q_{food,transp,i \rightarrow j,sell}^D + p_{cattle,transp,i \rightarrow j,sell} q_{cattle,transp,i \rightarrow j,sell}^D \nonumber \\
- \sum_{food} p_{food,transp,i \rightarrow j,buy} q_{food,transp,i \rightarrow j,buy}^D - \sum_{food} p_{food,transfer,i \rightarrow j,buy} q_{food,transfer,i \rightarrow j,buy}^D \nonumber \\
- p_{cattle,transp,i \rightarrow j,buy} q_{cattle,transp,i \rightarrow j,buy}^D \\
C = \left(C_{ij,fuel} + C_{ij,labour}\right)\left( q_{food,transp,i\rightarrow j,buy}^D + q_{food,transfer,i\rightarrow j,buy}^D \right. \nonumber \\ 
\left. + m_{cow} q_{cattle,transp,i\rightarrow j,buy}^D\right)  + p_{transp,capacity} \Delta q_{transp,capacity} \\
q_{food,transp,i \rightarrow j,sell}^D \leq \left(1 - r_{loss,food,distribution}\right) \left(q_{food,transp,i \rightarrow j,buy}^D + q_{food,transfer,i \rightarrow j,buy}^D \right) \label{transp eff con} \\
\sum_{food} \left( q_{food,transp,i \rightarrow j,buy}^D + q_{food,transfer,i \rightarrow j,buy}^D \right) + m_{cow} q_{cattle,transp,i \rightarrow j,buy}^D \leq q_{transp,capacity}
\label{capacity limit con} \\
\Delta q_{transp,capacity} \leq \Delta q_{transp,capacity,max} \label{capacity exp con}\\
q_{cattle,transp,i \rightarrow j,sell}^D - q_{cattle,transp,i \rightarrow j,buy}^D \leq 0 \label{cattle transp con}
\end{gather}

\noindent with decision variables 

\begin{align}
q_{food,transp,i \rightarrow j,buy}^D,q_{cattle,transp,i \rightarrow j,buy}^D,q_{cattle,transp,i \rightarrow j,sell}^D, \nonumber \\
q_{food,transfer,i \rightarrow j,buy}^D,q_{food,transp,i \rightarrow j,sell}^D,\Delta q_{transp,capacity}
\end{align}

\noindent and state variable

\begin{equation}
q_{transp,capacity}
\end{equation}

Assume that if food isn't transported from one node to another that it's consumed locally (i.e. no transportation cost or waste).

Revenue comes from transporting food commodities from one node to another, and the costs, aside from purchasing food in the source node, are for labour, fuel, and capacity expansion. (\ref{transp eff con}) accounts for food loss/spoilage during transport.  (\ref{capacity limit con}) enforces the capacity limit on distribution.  (\ref{capacity exp con}) limits the amount of expansion that can happen each month.  (\ref{cattle transp con}) ensures that the distributor doesn't sell more cattle than they buy.

The equilibrium conditions which have not yet been listed are

\begin{align}
q_{food,transp,i \rightarrow j,sell}^D - q_{food,transp,i \rightarrow j,sell}^S = 0 \\
q_{food,transfer,i \rightarrow j,buy}^D - q_{food,transfer,i \rightarrow j,buy}^S = 0
\end{align}

The duals of these equations give $p_{food,transp,i \rightarrow j,sell}$ and $p_{cattle,transp,i \rightarrow j,sell}$, respectively.

In future development, include an additional, non-geographic node as an export/import node.  This node will buy and sell cattle and food commodities from the Ethiopian nodes at prices and/or (maximum) quantities which will be determined exogenously in any given scenario, and it will not include transportation costs.  For example, to capture \textit{hajj} demand for meat, the import/export transporter will specify the total amount of beef/cattle that it wants and purchase it at the market price in each node so as to minimize its costs; alternatively, it sets the price it's willing to pay for beef and buys as much as Ethiopians will sell at that price.  As another example, food aid of some kind is supplied to a given region in a set amount, and this comes as free supplies (or supplies at a set price) to retail/storage.  We will specify the distribution model for this node in scenario development.


\subsection{Analysis}

Measure transportation efficiency between nodes $i$ and $j$ by

\begin{equation}
\eta_{ij} = \frac{d_{ij}}{v_c t_{ij}}
\end{equation}

\noindent where $d_{ij}$ is the distance between the node centres, $v_c$ is a characteristic travel velocity between the two points, and $t_{ij}$ is the time it takes to travel from one node centre to the other; transportation efficiency is implicitly a function of road quality and access.  Let us further assume that we can break down $\eta_{ij}$ into components directly attributable to each node:

\begin{equation}
\eta_{ij} = \frac{\eta_i + \eta_j}{2}
\end{equation}

Then

\begin{equation}
t_{ij} = \frac{2d_{ij}}{v_c \left(\eta_i + \eta_j\right)}
\end{equation}

There is also a distribution time within each node.  Let us assume a similar relationship between efficiency, a characteristic travel speed, and a linear measure of the size of the node.  Then

\begin{equation}
t_i = \frac{\sqrt{A_i}}{v_{c,internal} \eta_i}
\end{equation}

\noindent where $A_i$ is the area of node $i$ and $v_{c,internal}$ is the characteristic speed within the node.

Assume that transportation includes collection within a node, transportation to another node, and then distribution within that node.  Further assume that fuel efficiency per mass of food transported is constant with respect to the mass transported.  Then

\begin{equation}
C_{ij,fuel} = p_{fuel} \eta_{fuel} \left(\frac{d_{ij}}{\eta_{ij}} + \frac{\sqrt{A_i}}{\eta_{i}} + \frac{\sqrt{A_j}}{\eta_{j}} \right)
\end{equation}

\noindent where $\eta_{fuel}$ is the fuel efficiency in volume fuel per distance, per mass of food transported, $p_{fuel}$ is the price of fuel (per unit volume); fuel efficiency could be modified to become a function of mass transported and/or characteristic velocity.

Assume that labour costs are simply a function of time.  Then

\begin{equation}
C_{ij,labour} = p_{labour} \left( \frac{r_{labour,i} + r_{labour,j}}{2}t_{ij} + r_{labour,i}t_i + r_{labour,j}t_j \right)
\end{equation}

\noindent where $p_{labour}$ is the employee (hourly) wage.

Let us also assume that there is some food loss/waste/spoilage involved in the transportation process.  Consider food loss/waste/spoilage as a kind of exponential decay process with a `half-life' at which point half of the food has typically gone bad.  Then

\begin{equation}
r_{loss,food,distribution}\left(t\right) = 1 - \exp \left( \frac{- t \ln 2}{\alpha_q \tau_{food}}\right)
\end{equation}

\noindent where $r_{loss,food}\left(t\right)$ is the fraction of food wasted, $\tau_{food}$ is the food's half-life, and $\alpha_q$ is a parameter which captures the storage conditions of the food -- well-sealed, refridgerated storage would have a large $\alpha_q$ value (indicating that food would keep better), and open storage with high temperature and humidity would have a small $\alpha_q$ value (indicating that food would go bad more quickly).  Assume that we have $q_{food}$ at time $t=0$.  Then

\begin{alignat}{3}
q_{food} \left(t_1\right) \quad &= \quad  \left(1 - r_{loss,food}\left(t_1\right)\right) q_{food} \quad &=& \quad  \exp \left( \frac{- t_1 \ln 2}{\alpha_q \tau_{food}}\right) q_{food} \\
q_{food} \left(t_1 + t_2\right) \quad &= \quad  \left(1 - r_{loss,food}\left(t_2\right)\right) q_{food} \left(t_1 \right) \quad &=& \quad  \left(1 - r_{loss,food}\left(t_2\right)\right)\left(1 - r_{loss,food}\left(t_1\right)\right) q_{food} \nonumber \\
\quad &= \quad  \exp \left( \frac{- \left(t_1 + t_2\right) \ln 2}{\alpha_q \tau_{food}}\right) q_{food} \quad &=& \quad  \left(1 - r_{loss,food}\left(t_1 + t_2 \right)\right) q_{food}  \label{food loss similarity eqn} \\
&\left(1-r_{loss}\left(t_1\right)\right)\left(1-r_{loss}\left(t_2\right)\right) \quad &=& \quad  \left(1-r_{loss}\left(t_1+t_2\right)\right)
\end{alignat}

This way of modelling food loss is convenient because, as shown in (\ref{food loss similarity eqn}), it only requires us to know how much food we start with and how much time has elapsed to determine the percentage of food that has spoiled -- we do not need to know when we first began to store it or how much we had at first.  A linear rate of loss would not have this property:

\begin{align}
r_{loss} = 1 - \alpha t \\
\left(1-r_{loss}\left(t_1\right)\right)\left(1-r_{loss}\left(t_2\right)\right) = \alpha^2 t_1 t_2 \neq \left(1-r_{loss}\left(t_1+t_2\right)\right) = 1 - \alpha \left(t_1 + t_2\right)
\end{align}

For food distribution, then, 

\begin{equation}
r_{loss,food,distribution} = 1 - \exp \left( \frac{- \left(t_{ij} + t_i + t_j\right) \ln 2}{\alpha_{q,distribution} \tau_{food}}\right)
\end{equation}


\section{Storage/Retail}

The optimization problem maximizes profit over a three-year period but only keeps the first year's results:

\begin{gather}
\max \sum_{t=1}^{12} \left(R - C\right) + \sum_{t=13}^{24} \delta \left(\tilde{R} - \tilde{C}\right) + \sum_{t=25}^{36} \delta^2 \left(\hat{R} - \hat{C}\right) \\
R = \sum_{food} \left(p_{food,market,i} q_{food,market,i}^S + \sum_j p_{food,transfer,i \rightarrow j,buy} q_{food,transfer,i \rightarrow j,buy}^S \right) \\
C = \sum_{food} \left( p_{food,store,i} q_{food,store,i}^S + \sum_k p_{food,transp,k \rightarrow i,sell} q_{food,transp,k \rightarrow i,sell}^S \right) \nonumber \\
+ \sum_{food} C_{food} Q_{food,i} r_{month} + \sum_{food} p_{storage,food,expansion} \Delta Q_{food,storage,i} \\
Q_{food,i} \leq Q_{food,i,max} \label{storage capacity con} \\
\Delta Q_{food,storage,i} \leq \Delta Q_{food,storage,i,max} \label{storage exp con}\\
Q_{food,i,n+1} = \left(1-r_{loss,food,storage}\right) Q_{food,i,n} + \sum_k q_{food,transp,k \rightarrow i,sell}^S \nonumber \\
- \sum_j q_{food,transfer,i \rightarrow j,buy}^S + q_{food,store,i}^S - q_{food,market,i,n}^S \label{storage quantity eqn} \\
r_{loss,food,storage} = 1 - \exp \left( \frac{- t_{month} r_{month} \ln 2}{\alpha_{q,storage} \tau_{food}}\right) \label{storage loss eqn}\\
Q_{food,max,i,n+1} = Q_{food,max,i,n} + \Delta Q_{food,storage,i,n+1} \label{storage capacity eqn}
\end{gather}

\noindent with decision variables

\begin{equation}
q_{food,market,i}^S, q_{food,transp,k \rightarrow i,sell}^S, q_{food,transfer,i \rightarrow j,buy}^S, q_{food,store,i}^S \Delta Q_{food,storage,i} 
\end{equation}

\noindent and state variables

\begin{equation}
Q_{food,max,i}, Q_{food,i}
\end{equation}

\noindent where $t_{month}$ is the number of hours in each month, $C_{food}$ is the cost of storing food for a (30-day) month, and $\alpha_{q,storage}$ is the average storage quality in the region.  The storage/retail profit function is the revenue due to selling stock on the market minus the cost of acquiring new stock and the cost of storing existing stock.  The quantities with tildes and hats on them indicate that these are calculated from advisory variables: the storage optimization is calculated based on a three-year moving horizon, but only the first year's worth of decision variables are kept.

(\ref{storage capacity con}) limits total food stored by the maximum capacity available.  (\ref{storage exp con}) limits monthly storage expansion.  (\ref{storage quantity eqn}) tracks the amount of food in storage; this includes food loss/spoilage (\ref{storage loss eqn}), food bought, and food sold.  (\ref{storage capacity eqn}) tracks the maximum storage capacity.

The remaining equilibrium equation is

\begin{equation}
q_{food,market,i}^S - \int N_{pop} q_{food,market,i}^A \left(I\right) \rho_I \left(I\right) dI = 0
\end{equation}

The dual of this equation gives $p_{food,market,i}$.  This equation's form is slightly different because $q_{food,market,i}^A$ is measured on an (average) individual basis and varies with income.  This also allows for population changes to directly impact demand.

\section{Access/Consumers}

Note that the food quantities are prices used here correspond to $p_{food,market,i}$ and $q_{food,market,i}^A$, respectively; the notation has been simplified for ease of following the derivations.  The food quantities are also an average individual measure, not a total measure.

\subsection{Utility Optimization}

Assume that utility $U$ is a function of food quantity $q$ and parameterized by income $I$: $U = U\left(q;I\right)$.  Further assume that the proportion of the population with income $I$ can be represented by the probability distribution function $\rho_I \left(I\right)$.  We can then calculate the average, or expected, utility $U_{ave}$ and produce a $q\left(I\right)$ which optimizes the utility function.  Assume that the utility function can be expressed in the form $U = \alpha q - \beta q^2$, where $\alpha$ and/or $\beta$ may be functions of $I$ (and implicitly of prices $p$).

\begin{align}
U_{ave} \left(q\right) = E\left[U\left(q\right)\right] \quad &= \quad  \int U \left(q;I\right) \rho_I \left(I\right) dI \\
\frac{d U_{ave}}{dq} \quad &= \quad  \int \frac{\partial U \left(q;I\right)}{\partial q} \rho_I \left(I\right) dI\\
 \quad &= \quad  0 \\
\int  \left(\alpha - 2 \beta q\right) \rho_I \left(I\right) dI \quad &= \quad  0 \\
q^*\left(I\right) \quad &= \quad  \frac{\alpha}{2 \beta} \\
E\left[q^*\right] \quad &= \quad  \int  \frac{\alpha}{2 \beta} \rho_I \left(I\right) dI \\
q_{total} \quad &= \quad  \int N_{pop} q^* \left(I\right) \rho_I \left(I\right) dI
\end{align}

If there are multiple food quantities $\mathbf{q}$, then $\mathbf{q}^* = \frac{1}{2} \left[\boldsymbol \beta\right]^{-1} \boldsymbol \alpha$.  $\alpha$ and $\beta$ will need to satisfy the Slutsky conditions.

Take income profiles and changes as exogenous variables.

\subsection{Utility Formulation}

Assume that we can specify the (inverse) demand curves for protein, calories, and an aggregate micronutrient measure:

\begin{align}
\mathbf{q_{nutrient}} \quad &= \quad  \left\{ \begin{array}{c}
q_{protein} \\
q_{calories} \\
q_{agg}
\end{array} \right\} \\
\mathbf{q_{nutrient}} \quad &= \quad  r_{month} \left(\mathbf{a}\left(I\right) - \mathbf{B} \mathbf{p_{nutrient}}\right) \\
\mathbf{p_{nutrient}} \quad &= \quad  \mathbf{B}^{-1} \mathbf{a} \left(I\right) - \frac{1}{r_{month}} \mathbf{B}^{-1} \mathbf{q_{nutrient}}
\end{align}

The demand is normalized to a 30-day month, so $r_{month}$ is used to account for increased/decreased consumption due to increased/decreased month length.  Assume that all three nutrients are normal goods ($\frac{d \mathbf{a}}{d I} > 0$), all three are ordinary goods ($B_{jj} > 0 \ \forall j$), and that protein and the aggregated micronutrient measure are complements ($B_{31},B_{13} > 0$).  Assume that demand grows proportionally to the square root of income:

\begin{equation}
\mathbf{a} = \sqrt{I} \tilde{\mathbf{a}}
\end{equation}

Alternatively, consider using $\log I$ instead of $\sqrt{I}$.

See strauss86col, smith86col for some data which might be useful for estimating the demand curves.

Assume that the nutritional contents of food per unit mass can be expressed in a matrix $\boldsymbol \gamma_{nutrient}$:

\begin{equation}
\mathbf{q_{nutrient}} = \boldsymbol \gamma_{nutrient} \mathbf{q_{food}}
\end{equation}

The utility function $U$ is then

\begin{align}
U = \int_0^{\mathbf{q_{nutrient}}} \mathbf{B}^{-1} \mathbf{a} \left(I\right) - \frac{1}{r_{month}} \mathbf{B}^{-1} \mathbf{q'_{nutrient}} d \mathbf{q'_{nutrient}} - \mathbf{p_{food}}^T \mathbf{q_{food}} \\
= \mathbf{q_{nutrient}}^T \mathbf{B}^{-1} \mathbf{a} \left(I\right) - \frac{1}{2 r_{month}} \mathbf{q_{nutrient}}^T \mathbf{B}^{-1} \mathbf{q_{nutrient}} - \mathbf{p_{food}}^T \mathbf{q_{food}} \\
= \mathbf{q_{food}}^T \boldsymbol \gamma_{nutrient}^T \mathbf{B}^{-1} \mathbf{a} \left(I\right) - \frac{1}{2 r_{month}} \mathbf{q_{food}}^T \boldsymbol \gamma_{nutrient}^T \mathbf{B}^{-1} \boldsymbol \gamma_{nutrient} \mathbf{q_{food}} - \mathbf{p_{food}}^T \mathbf{q_{food}} 
\end{align}

Note that these quantities and utilities are functions of $I$ (but prices are constant across $I$).  To get the total quantities for a particular income group -- $I \in \left[I_1,I_2\right]$ -- integrate over 

\begin{equation}
N_{pop} \int \limits_{I_1}^{I_2} \mathbf{q} \left(I\right) \rho_I \left(I\right) dI
\end{equation}

Consider adding a quadratic penalty term to represent consumers' dietary conservatism (i.e. they are disinclined to change what they eat):

\begin{equation}
\frac{1}{2} p_{change} \left\|\frac{\mathbf{q}_{food,n}}{r_{month,n}} - \frac{\mathbf{q}_{food,n-1}}{r_{month,n-1}}\right\|^2
\end{equation}

\noindent where $\mathbf{q_{food,n}}$ is the consumption profile from time step $n$.  Note the normalization with $r_{month}$ to account for differences in month length.

\subsection{Rolling Time Horizon}

In order to make the rolling time horizon work for livestock production and storage, we use

\begin{gather}
U \left(t\right) = \mathbf{q_{food,t}}^T \boldsymbol \gamma_{nutrient}^T \mathbf{B}^{-1} \mathbf{a} \left(I\right) - \frac{1}{2r_{month}} \mathbf{q_{food,t}}^T \boldsymbol \gamma_{nutrient}^T \mathbf{B}^{-1} \boldsymbol \gamma_{nutrient} \mathbf{q_{food,t}} \nonumber \\ 
- \mathbf{p_{food,t}}^T \mathbf{q_{food,t}} - \frac{1}{2} p_{change} \left\|\frac{\mathbf{q}_{food,t}}{r_{month,t}} - \frac{\mathbf{q}_{food,t-1}}{r_{month,t-1}}\right\|^2 \\
\max \sum_{t=1}^{12} U\left(t\right) + \sum_{t=13}^{24} \delta \tilde{U}\left(t\right) + \sum_{t=25}^{36} \delta^2 \hat{U}\left(t\right)
\end{gather}

\noindent and keep only the first year's results.

\subsection{Further Disaggregation}

Let us further consider consumption broken down by age.  Assume that there is a joint population density function of both income, $I$, and age $a$: $\rho \left(I,a\right)$.  We can express this using the marginal distribution with respect to income (which we used previously to calculate consumption):

\begin{equation}
\rho \left(I,a\right) = \rho_{a|I} \left(a|I\right) \rho_I \left(I\right)
\end{equation}

We will also require a function $w\left(a\right)$ to account for the difference in consumption between different ages -- $w$ is average consumption as a fraction of an average adult baseline (e.g. for a three-year old eating 20\% as much as an adult, $w\left(3\right) = .2$).  Then

\begin{align}
q \left(I,a\right) \quad &= \quad  \frac{w\left(a\right)}{\mu_{w|I}} q\left(I\right) \\
\mu_{w|I} \quad &= \quad  \int w\left(a\right) \rho_{a|I} \left(a|I\right) da
\end{align}

Note that $w$ is assumed to be independent of income.

We could similarly include gender in our disaggregation much as we did with age.  We could then have one of several density functions

\begin{align}
\rho \left(g,a,I\right) \quad &= \quad  \rho_{a,g} \left(a,g|I\right) \rho_I \left(I\right) \\
\rho \left(g,a,I\right) \quad &= \quad  \rho_g \left(g|a,I\right) \rho_a \left(a|I\right) \rho_I \left(I\right) \\
\rho \left(g,a,I\right) \quad &= \quad  \rho_g \left(g\right) \rho_a \left(a\right) \rho_I \left(I\right)
\end{align}

\noindent where $g$ is gender.  Our weighting function would then be $w \left(a,g\right)$, and

\begin{align}
\mu_{w|I} \quad &= \quad  \int w\left(a,g\right) \rho_{a,g} \left(a,g|I\right) \left(a|I\right) da dg \\
q \left(I,a,g\right) \quad &= \quad  \frac{w\left(a,g\right)}{\mu_{w|I}} q\left(I\right)
\end{align}

The integral in question would in fact end up being a summation over the two different values of $g$, and different density representations, as shown above, could be used instead of $\rho_{a,g} \left(a,g|I\right)$.  

\section{Utilization}

Assume that there exists a certain amount of food waste, $r_{loss,food,utilization}$ within the household.  Then

\begin{equation}
q_{food,i}^U = \left(1-r_{loss,food,utilization}\right) q_{food,market,i}^A
\end{equation}

We could further assume that nutrient intake is reduced by sanitation conditions and GI disease rates, and let the population incidence rate of diarrhea in the last month act as a proxy for these sanitation conditions and GI disease rates.  Then

\begin{align}
\boldsymbol \nu_{actual} \quad &= \quad  r_{health,utilization} \boldsymbol \nu \left(q_{food,i}^U\right) \\
r_{health,utilization} \quad &= \quad  1 - r_{diarrhea} \rho_{diarrhea}
\end{align}

\noindent where $r_{diarrhea}$ is a constant reflecting the impact of diarrhea on nutrient uptake and $\rho_{diarrhea}$ is the population incidence rate over the last month.  Note that incidence rate may be broken down by age, gender, and per capita income if so desired:

\begin{align}
r_{health,utilization} \left(a,g,I\right) \quad &= \quad  1 - r_{diarrhea} \rho_{diarrhea}\left(a,g,I\right) \\
\boldsymbol \nu_{actual} \left(a,g,I\right) \quad &= \quad  r_{health,utilization}\left(a,g,I\right) \boldsymbol \nu \left( q_{food,i}^U \left(a,g,I\right) \right) \\
\boldsymbol \nu_{working} \quad &= \quad  \int \boldsymbol \nu_{actual} \left(a \geq 16,I\right) \rho_I \left(I\right) \rho_I dI \label{nu ave eqn}
\end{align}

(\ref{nu ave eqn}) is the average nutrient consumption for adults, across income levels, for a given node.

However, modelling nutrient absorption is very difficult and fraught with complications -- this simple approach likely will not work -- so we will ignore it for the time being.

\section{Food Security}

Consider diet diversity, fraction of income used on food, and food price volatility as measures of food security.  Measure diet diversity $\sigma_{div}$ using

\begin{equation}
\sigma_{div} = \frac{\sum \limits_{group} q_{food,group} - \max \limits_{group} q_{food,group}}{\max \limits_{group} q_{food,group}} \frac{n_{group}}{n_{food,group} - 1}
\end{equation}

\noindent where $n_{food}$ is the number of food commodities being considered and food commodities are grouped by type (e.g. cereals, pulses, meat, etc.).  $\sigma_{div}$ is 0 when only one food group is being consumed and 1 when all food groups are being consumed equally.  Use the FANTA guidelines on dietary diversity in demarcating groups.

The ratio of income spent on food to total income, $\sigma_{inc}$, is

\begin{equation}
\sigma_{inc} = \frac{q^*\left(I\right) p_{food}}{I} 
\end{equation}

The food price volatility, $\sigma_{vol}$, is

\begin{equation}
\sigma_{vol} = \sqrt{\sum_{food} \left(\frac{\sigma_{p,food}}{\mu_{p,food}}\right)^2}
\end{equation}

\noindent where $\sigma_{p,food}$ is the standard deviation of the food commodity's price over the last year (or a different time length, if so desired), and $\mu_{p,food}$ is the mean of the food commodity's price over the same time period.


\section{MCP formulation}

Note that equations connecting montly timesteps will require special handling -- particularly with the advisory variables.  We have not explicitly spelled out how this is done, but it should be apparent from what is given.  Also, substitute hats and tildes, and add $\delta$'s where appropriate for the advisory variable KKT conditions.

\subsection{Crop Producer}

The optimization problem for each region is

\begin{gather}
\max U = \sum_{t=1}^{12} R\left(t\right) - C - C_{change} + \delta \left[\sum_{t=12}^{24} \tilde{R}\left(t\right) - \tilde{C} - \tilde{C}_{change}\right] \nonumber \\
+ \delta^2 \left[\sum_{t=25}^{36} \hat{R}\left(t\right) - \hat{C} - \hat{C}_{change}\right] \\
g_1 = \Delta A_{conv} - \Delta A_{conv,max} \leq 0\\
g_2 = q_{crop,store,i}^C + \sum_j q_{crop,transp,i \rightarrow j,buy}^C - \sum_{season} m_{crop,season}  A_{crop,season} r_{sq} r_{sf} \leq 0\\
g_3 = \sum_{crop,season} A_{crop,season} - A_{tot} \leq 0 \\
h_{1,n} = A_{tot,n} - A_{tot,n-1} - \Delta A_{conv,n} = 0
\end{gather}

\noindent with decision variables

\begin{equation}
A_{crop,season},\Delta A_{conv},q_{crop,store,i}^C,q_{crop,transp,i\rightarrow j,buy}^C
\end{equation}

\noindent and state variable

\begin{equation}
A_{tot}
\end{equation}

\noindent where

\begin{gather}
R\left(t\right) = \sum_{crop} p_{crop,store,i} q_{crop,store,i}^C + \sum_{crop} \sum_j p_{crop,transp,i\rightarrow j,buy} q_{crop,transp,i\rightarrow j,buy}^C \\
C = r_{labour,ave} p_{labour} \sum_{crop,season} A_{crop,season} t_{crop} \nonumber \\
+ p_{fuel} \sum_{crop,season} A_{crop,season} q_{fuel,crop} + p_{conv} \Delta A_{conv} \\
C_{change,n} = \frac{1}{2} \sum_{crop,season} p_{change,crop} \left( A_{crop,season,n} - A_{crop,season,n-1} \right)^2
\end{gather}

The relevant pieces of derivative information are

\begin{gather}
\frac{\partial U}{\partial A_{crop,season,n}} = - \left(r_{labour,ave} t_{crop} p_{labour} + p_{fuel} q_{fuel,crop}\right) \nonumber \\
- p_{change} \left(A_{crop,season,n} - A_{crop,season,n-1}\right) \nonumber \\
+ p_{change} \left(A_{crop,season,n+1} - A_{crop,season,n}\right) \\
\frac{\partial g_2}{\partial A_{crop,season}} = - m_{crop,season} r_{sq} r_{sf} \\
\frac{\partial g_3}{\partial A_{crop,season}} = 1\\ 
\nonumber \\
\frac{\partial U}{\partial \Delta A_{conv}} = -p_{conv} \\
\frac{\partial g_1}{\partial \Delta A_{conv}} = 1 \\
\frac{\partial h_1}{\partial \Delta A_{conv}} = -1 \\
\nonumber \\
\frac{\partial U}{\partial q_{crop,store,i}^C} = p_{crop,store,i} \\
\frac{\partial g_2}{\partial q_{crop,store,i}^C} = 1 \\
\nonumber \\
\frac{\partial U}{\partial q_{crop,transp,i\rightarrow j,buy}^C} = p_{crop,transp,i \rightarrow j,buy} \\
\frac{\partial g_2}{\partial q_{crop,transp,i\rightarrow j,buy}^C} = 1 \\
\nonumber \\
\frac{\partial g_3}{\partial A_{tot}} = -1 \\
\frac{\partial h_{1,n}}{\partial A_{tot,n}} = 1 \\
\frac{\partial h_{1,n}}{\partial A_{tot,n-1}} = -1
\end{gather}

The MCP equations to solve are

\begin{gather}
\frac{\partial U}{\partial A_{crop,season}} - \sum_t p_{crop} \frac{\partial g_2}{\partial A_{crop,season}} - p_{land} \frac{\partial g_3}{\partial A_{crop,season}} = 0 \\
\frac{\partial U}{\partial \Delta A_{conv}} - p_{exp,max} \frac{\partial g_1}{\partial \Delta A_{conv}} - p_{land,exp} \frac{\partial h_1}{\partial \Delta A_{conv}} = 0 \\
\frac{\partial U}{\partial q_{crop,store,i}^C} - p_{crop} \frac{\partial g_2}{\partial q_{crop,store,i}^C} = 0 \\
\frac{\partial U}{\partial q_{crop,transp,i\rightarrow j,buy}^C} - p_{crop} \frac{\partial g_2}{\partial q_{crop,transp,i\rightarrow j,buy}^C} = 0 \\
- p_{land,n} \frac{\partial g_{3,n}}{\partial A_{tot,n}} - p_{land,exp,n} \frac{\partial h_{1,n}}{\partial A_{tot,n}} - p_{land,exp,n+1} \frac{\partial h_{1,n+1}}{\partial A_{tot,n}} = 0 \\
0 \leq -g_1 \perp p_{exp,max} \geq 0 \\
0 \leq -g_2 \perp p_{crop} \geq 0 \\
0 \leq -g_3 \perp p_{land} \geq 0 \\
q_{crop,store,i}^C - q_{crop,store,i}^S = 0 \ , \ p_{crop,store,i} \ \text{free} \\
q_{crop,transp,i\rightarrow j,buy}^C - q_{crop,transp,i\rightarrow j,buy}^D = 0 \ , \ p_{crop,transp,i\rightarrow j,buy} \ \text{free} \\
h_1 = 0 \ , \ p_{land,exp} \ \text{free}
\end{gather}

Any of the equations with commodity flows in them have to be solved at each time step.  All commodity flows have to be greater than or equal to 0, too.

$U$, $g_1$ and $g_4$ are solved once for each year.  $g_2$ is solved every month.  Note also that

\begin{equation}
m_{crop} = \left\{ \begin{array}{cc}
m_h			& \text{in its harvest month} \\
0				& \text{else} 
\end{array} \right.
\end{equation}

\subsection{Livestock Producer}

The optimization for each region each month is

\begin{gather}
\max U = \sum_{t=1}^{12} \left(R - C\right) + \sum_{t=13}^{24} \delta \left(\tilde{R} - \tilde{C}\right) + \sum_{t=25}^{36} \delta^2 \left(\hat{R} - \hat{C}\right) + p_{cattle,ave} N_{cattle,i,final} \\
g_1 = q_{milk,store,i}^L + \sum_j q_{milk,transp,i \rightarrow j,buy}^L - \eta_{production} r_{dairy} N_{cattle,i} \mu_{production} r_{month} \leq 0\\
g_2 = - N_{cattle,i} \leq 0 \\
h_{1,n} = N_{cattle,i,n} - \left(1 + k - \kappa \right) N_{cattle,i,n-1} + \frac{q_{beef,store,i,n}^L}{r_{meat} m_{cow}} \nonumber \\
+ \sum_j \left( q_{cattle,transp,i \rightarrow j,buy,n}^L + \frac{q_{beef,transp,i \rightarrow j,buy,n}^L}{r_{meat} m_{cow}} - q_{cattle,transp,j\rightarrow i,sell,n} \right)= 0
\end{gather}

\noindent with decision variables

\begin{gather}
q_{milk,store,i}^L,q_{beef,store,i}^L,q_{milk,transp,i \rightarrow j}^L,q_{cattle,transp,i \rightarrow j,buy}^L, \nonumber \\
q_{beef,transp,i \rightarrow j}^L,q_{cattle,transp,j \rightarrow i,sell}^L, 
\end{gather}

\noindent and state variable

\begin{equation}
N_{cattle,i}
\end{equation}

\noindent where

\begin{gather}
R = p_{milk,store,i} q_{milk,store,i}^L + p_{beef,store,i} q_{beef,store,i}^L \nonumber \\
+ \sum_j \left( p_{milk,transp,i \rightarrow j,buy} q_{milk,transp,i \rightarrow j,buy}^L + p_{cattle,transp,i \rightarrow j,buy} q_{cattle,transp,i \rightarrow j,buy}^L \right. \\
\left. + p_{beef,transp,i \rightarrow j,buy} q_{beef,transp,i \rightarrow j,buy}^L \right) + \left(q_{beef,store,i}^L + \sum_j q_{beef,transp,i \rightarrow j,buy}^L \right) \frac{p_{hide}}{m_{cow} r_{meat}} \\
C = \sum_j p_{cattle,transp,j \rightarrow i,sell} q_{cattle,transp,j \rightarrow i,sell}^L \nonumber \\
+ \left(r_{feed} \mu_{feed} p_{feed} + r_{labour} p_{labour} t_{cattle} \right) N_{cattle,i} \\
\eta_{production} = 1- \alpha_{dairy} \sum_t \tau_t \\
\tau_t = \max \left\{\left[1.8 \left(0.35 T_{db} + 0.65 T_{wb}\right) + 32\right] - 74,0\right\}
\end{gather}


The relevant pieces of derivative information are

\begin{gather}
\frac{\partial U}{\partial q_{milk,store,i}^L} = p_{milk,store,i} \\
\frac{\partial g_1}{\partial q_{milk,store,i}^L} = 1 \\
\frac{\partial U}{\partial q_{milk,transp,i \rightarrow j,buy}^L} = p_{milk,transp,i \rightarrow j,buy} \\
\frac{\partial g_1}{\partial q_{milk,transp,i \rightarrow j,buy}^L} = 1 \\
\frac{\partial U}{\partial q_{beef,store,i}^L} = p_{beef,store,i} + p_{hide} \\
\frac{\partial h_1}{\partial q_{beef,store,i}^L} = \frac{1}{r_{meat} m_{cow}} \\
\frac{\partial U}{\partial q_{beef,transp,i \rightarrow j,buy}^L} = p_{beef,transp,i \rightarrow j,buy} + \frac{p_{hide}}{m_{cow} r_{meat}} \\
\frac{\partial h_1}{\partial q_{beef,transp,i \rightarrow j,buy}^L} = \frac{1}{r_{meat} m_{cow}} \\
\frac{\partial U}{\partial q_{cattle,transp,i \rightarrow j,buy}^L} = p_{cattle,transp,i \rightarrow j,buy} \\
\frac{\partial h_1}{\partial q_{cattle,transp,i \rightarrow j,buy}^L} = 1 \\
\frac{\partial U}{\partial q_{cattle,transp,j \rightarrow i,sell}^L} = - p_{cattle,transp,j \rightarrow i,sell} \\
\frac{\partial h_1}{\partial q_{cattle,transp,j \rightarrow i,sell}^L} = -1 \\
\frac{\partial U}{\partial N_{cattle,i}} = - \left(r_{feed} \mu_{feed} p_{feed} + r_{labour} p_{labour} t_{cattle}\right) \ \left(+ p_{cattle,ave} \text{ for the final } N_{cattle,i}\right) \\ 
\frac{\partial g_1}{\partial N_{cattle,i}} = - \eta_{production} r_{dairy} \mu_{production} r_{month} \\
\frac{\partial g_2}{\partial N_{cattle,i}} = -1 \\
\frac{\partial h_{1,n}}{\partial N_{cattle,i,n}} = 1 \\
\frac{\partial h_{1,n+1}}{\partial N_{cattle,i,n}} = - \left(1 + k - \kappa \right)
\end{gather}


The MCP equations to solve are

\begin{gather}
\frac{\partial U}{\partial q_{milk,store,i}^L} - p_{milk,i} \frac{\partial g_1}{\partial q_{milk,store,i}^L} = 0\\
\frac{\partial U}{\partial q_{milk,transp,i \rightarrow j,buy}^L} - p_{milk,i} \frac{\partial g_1}{\partial q_{milk,transp,i \rightarrow j,buy}^L} = 0\\
\frac{\partial U}{\partial q_{beef,store,i}^L} - p_{cattle,i} \frac{\partial h_1}{\partial q_{beef,store,i}^L} = 0\\
\frac{\partial U}{\partial q_{beef,transp,i \rightarrow j,buy}^L} - p_{cattle,i} \frac{\partial h_1}{\partial q_{beef,transp,i \rightarrow j,buy}^L} = 0\\
\frac{\partial U}{\partial q_{cattle,transp,i \rightarrow j,buy}^L} - p_{cattle,i} \frac{\partial h_1}{\partial q_{cattle,transp,i \rightarrow j,buy}^L} = 0 \\
\frac{\partial U}{\partial q_{cattle,transp,j \rightarrow i,sell}^L} - p_{cattle,i} \frac{\partial h_1}{\partial q_{cattle,transp,j \rightarrow i,sell}^L}  = 0 \\
\frac{\partial U}{\partial N_{cattle,i,n}} - p_{milk,i,n} \frac{\partial g_{1,n}}{\partial N_{cattle,i,n}} - p_{herd,n} \frac{\partial g_{2,n}}{\partial N_{cattle,i,n}} \nonumber \\
- p_{cattle,i,n+1} \frac{\partial h_{1,n+1}}{\partial N_{cattle,i,n}} - p_{cattle,i,n} \frac{\partial h_{1,n}}{\partial N_{cattle,i,n}} = 0\\
0 \leq -g_1 \perp p_{milk,i} \geq 0 \\
0 \leq -g_2 \perp p_{herd} \geq 0 \\
q_{milk,store,i}^L - q_{milk,store,i}^S = 0 \ , \ p_{milk,store,i} \ \text{free} \\
q_{beef,store,i}^L - q_{beef,store,i}^S = 0 \ , \ p_{beef,store,i} \ \text{free} \\
q_{milk,transp,i \rightarrow j,buy}^L - q_{milk,transp,i \rightarrow j,buy}^D = 0 \ , \ p_{milk,transp,i \rightarrow j,buy} \ \text{free} \\
q_{beef,transp,i \rightarrow j,buy}^L - q_{beef,transp,i \rightarrow j,buy}^D = 0 \ , \ p_{beef,transp,i \rightarrow j,buy} \ \text{free} \\
q_{cattle,transp,i \rightarrow j,buy}^L - q_{cattle,transp,i \rightarrow j,buy}^D = 0 \ , \ p_{cattle,transp,i \rightarrow j,buy} \ \text{free} \\
q_{cattle,transp,j \rightarrow i,sell}^L - q_{cattle,transp,j \rightarrow i,sell}^D = 0 \ , \ p_{cattle,transp,j \rightarrow i,sell} \ \text{free} \\
h_1 = 0 \ , \ p_{cattle,i} \ \text{free}
\end{gather}

\subsection{Distributor}

The optimization for each region is

\begin{gather}
\max U = \sum_{t=1}^{12} \left(R - C\right) + \sum_{t=13}^{24} \delta \left(\tilde{R} - \tilde{C}\right) + \sum_{t=25}^{36} \delta^2 \left(\hat{R} - \hat{C}\right) \\
g_1 = q_{food,transp,i \rightarrow j,sell}^D - \left(1 - r_{loss,food}\right) \left(q_{food,transp,i \rightarrow j,buy}^D + q_{food,transfer,i \rightarrow j,buy}^D \right) \leq 0 \\
g_2 = \sum_{food} \left( q_{food,transp,i \rightarrow j,buy}^D + q_{food,transfer,i \rightarrow j,buy}^D \right) + m_{cow} q_{cattle,transp,i \rightarrow j,buy}^D \nonumber \\ 
- q_{transp,capacity} \leq 0 \\
g_3 = \Delta q_{transp,capacity} - \Delta q_{transp,capacity,max} \leq 0 \\
g_4 = q_{cattle,transp,i \rightarrow j,sell}^D - q_{cattle,transp,i \rightarrow j,buy}^D \leq 0 \\
h_{1,n} = q_{transp,capacity,n} - q_{transp,capacity,n-1} - \Delta q_{transp,capacity,n} = 0 \\
\end{gather}

\noindent with decision variables

\begin{gather}
q_{food,transp,i \rightarrow j,buy}^D,q_{cattle,transp,i \rightarrow j,buy}^D,q_{cattle,transp,i \rightarrow j,sell}^D, \nonumber \\
q_{food,transfer,i \rightarrow j,buy}^D,q_{food,transp,i \rightarrow j,sell}^D,\Delta q_{transp,capacity} 
\end{gather}

\noindent and state variables

\begin{equation}
q_{transp,capacity}, C_{ij,labour}
\end{equation}

\noindent where

\begin{gather}
R = \sum_{food} p_{food,transp,i\rightarrow j,sell} q_{food,transp,i \rightarrow j,sell}^D + p_{cattle,transp,i \rightarrow j,sell} q_{cattle,transp,i \rightarrow j,sell}^D \nonumber \\
- \sum_{food} p_{food,transp,i \rightarrow j,buy} q_{food,transp,i \rightarrow j,buy}^D - \sum_{food} p_{food,transfer,i \rightarrow j,buy} q_{food,transfer,i \rightarrow j,buy}^D \nonumber \\
- p_{cattle,transp,i \rightarrow j,buy} q_{cattle,transp,i \rightarrow j,buy}^D \\
C = \left(C_{ij,fuel} + C_{ij,labour}\right)\left( q_{food,transp,i\rightarrow j,buy}^D + q_{food,transfer,i\rightarrow j,buy}^D + m_{cow} q_{cattle,transp,i\rightarrow j,buy}^D\right) \nonumber \\ + p_{transp,capacity} \Delta q_{transp,capacity} \\
C_{ij,fuel} = p_{fuel} \eta_{fuel} \left(\frac{2d_{ij}}{\eta_i + \eta_j} + \frac{\sqrt{A_i}}{\eta_{i}} + \frac{\sqrt{A_j}}{\eta_{j}} \right) \\
C_{ij,labour} = p_{labour} \mu_{transp} \left( \frac{r_{labour,i} + r_{labour,j}}{2}t_{ij} + r_{labour,i}t_i + r_{labour,j}t_j \right) \\
t_{ij} = \frac{2d_{ij}}{v_c \left(\eta_i + \eta_j\right)} \\
t_i = \frac{\sqrt{A_i}}{v_{c,internal} \eta_i} \\
r_{loss,food,distribution} = 1 - \exp \left( \frac{- \left(t_{ij} + t_i + t_j\right) \ln 2}{\alpha_{q,distribution} \tau_{food}}\right)
\end{gather}

The relevant pieces of derivative information are

\begin{gather}
\frac{\partial U}{\partial q_{food,transp,i \rightarrow j,buy}^D} = - p_{food,transp,i \rightarrow j,buy} - \left( C_{ij,fuel} + C_{ij,labour}\right) \\
\frac{\partial g_1}{\partial q_{food,transp,i \rightarrow j,buy}^D} = - \left( 1 - r_{loss,food,distribution}\right) \\
\frac{\partial g_2}{\partial q_{food,transp,i \rightarrow j,buy}^D} = 1 \\
\frac{\partial U}{\partial q_{food,transp,i \rightarrow j,sell}^D} = p_{food,transp,i \rightarrow j,sell} \\
\frac{\partial g_1}{\partial q_{food,transp,i \rightarrow j,buy}^D} = 1 \\
\frac{\partial U}{\partial q_{cattle,transp,i \rightarrow j,buy}^D} = - p_{cattle,transp,i \rightarrow j,buy} - m_{cow} \left( C_{ij,fuel} + C_{ij,labour}\right) \\
\frac{\partial g_2}{\partial q_{cattle,transp,i \rightarrow j,buy}^D} = m_{cow} \\
\frac{\partial g_4}{\partial q_{cattle,transp,i \rightarrow j,buy}^D} = -1 \\
\frac{\partial U}{\partial q_{cattle,transp,i \rightarrow j,sell}^D} = p_{cattle,transp,i \rightarrow j,sell} \\
\frac{\partial g_4}{\partial q_{cattle,transp,i \rightarrow j,sell}^D} = 1 \\
\frac{\partial U}{\partial q_{food,transfer,i \rightarrow j,buy}^D} = - p_{food,transfer,i \rightarrow j,buy} - \left( C_{ij,fuel} + C_{ij,labour}\right) \\
\frac{\partial g_1}{\partial q_{food,transfer,i \rightarrow j,buy}^D} = - \left( 1 - r_{loss,food,distribution}\right) \\
\frac{\partial g_2}{\partial q_{food,transfer,i \rightarrow j,buy}^D} = 1 \\
\frac{\partial U}{\partial \Delta q_{transp,capacity}} = - p_{transp,capacity} \\
\frac{\partial g_3}{\partial \Delta q_{transp,capacity}} = 1 \\
\frac{\partial h_1}{\partial \Delta q_{transp,capacity}} = -1 \\
\frac{\partial g_2}{\partial q_{transp,capacity,n}} = -1 \\
\frac{\partial h_{1,n}}{\partial q_{transp,capacity,n}} = 1 \\
\frac{\partial h_{1,n}}{\partial q_{transp,capacity,n-1}} = -1 \\
\end{gather}

The MCP equations to solve are

\begin{gather}
\frac{\partial U}{\partial q_{food,transp,i \rightarrow j,buy}^D} - p_{transp,eff} \frac{\partial g_1}{\partial q_{food,transp,i \rightarrow j,buy}^D} - p_{transp} \frac{\partial g_2}{\partial q_{food,transp,i \rightarrow j,buy}^D} = 0 \\ 
\frac{\partial U}{\partial q_{food,transp,i \rightarrow j,sell}^D} - p_{transp,eff} \frac{\partial g_1}{\partial q_{food,transp,i \rightarrow j,sell}^D} = 0 \\ 
\frac{\partial U}{\partial q_{food,transfer,i \rightarrow j,buy}^D} - p_{transp,eff} \frac{\partial g_1}{\partial q_{food,transfer,i \rightarrow j,buy}^D} - p_{transp} \frac{\partial g_2}{\partial q_{food,transfer,i \rightarrow j,buy}^D} = 0 \\
\frac{\partial U}{\partial q_{cattle,transp,i \rightarrow j,buy}^D} - p_{transp} \frac{\partial g_2}{\partial q_{transp,cattle,i \rightarrow j,buy}^D} - p_{cattle,transp} \frac{\partial g_4}{\partial q_{cattle,transp,i \rightarrow j,buy}^D} = 0 \\
\frac{\partial U}{\partial q_{cattle,transp,i \rightarrow j,sell}^D} - p_{cattle,transp} \frac{\partial g_4}{\partial q_{transp,cattle,i \rightarrow j,sell}^D} = 0 \\ 
\frac{\partial U}{\partial \Delta q_{transp,capacity}} - p_{transp,exp,max} \frac{\partial g_3}{\partial \Delta q_{transp,capacity}} - p_{transp,exp} \frac{\partial h_1}{\partial \Delta q_{transp,capacity}} = 0 \\
-p_{transp,n}\frac{\partial g_{2,n}}{\partial q_{transp,capacity,n}} - p_{transp,exp,n} \frac{\partial h_{1,n}}{\partial q_{transp,capacity,n}} - p_{transp,exp,n+1} \frac{\partial h_{1,n+1}}{\partial q_{transp,capacity,n}} = 0 \\
0 \leq -g_1 \perp p_{transp,eff} \geq 0 \\
0 \leq -g_2 \perp p_{transp} \geq 0 \\
0 \leq -g_3 \perp p_{transp,exp,max} \geq 0 \\
0 \leq -g_4 \perp p_{transp,cattle} \geq 0 \\
q_{food,transfer,i \rightarrow j,buy}^D - q_{food,transfer,i \rightarrow j,buy}^S = 0 \ , \ p_{food,transfer,i \rightarrow j,buy} \ \text{free} \\
q_{food,transp,j \rightarrow i,sell}^D - q_{food,transp,j \rightarrow i,sell}^S = 0 \ , \ p_{food,transp,j \rightarrow i,sell} \ \text{free} \\
h_1 = 0 \ , \ p_{transp,exp} \ \text{free}
\end{gather}


\subsection{Storage}

The optimization for each region each month is

\begin{gather}
\max U = \sum_{t=1}^{12} \left(R - C\right) + \sum_{t=13}^{24} \delta \left(\tilde{R} - \tilde{C}\right) + \sum_{t=25}^{36} \delta^2 \left(\hat{R} - \hat{C}\right) \\
g_1 = Q_{food,i} - Q_{food,i,max} \leq 0 \\
g_2 = \Delta Q_{food,storage,i} - \Delta Q_{food,storage,i,max} \leq 0 \\
g_3 = - Q_{food,i} \leq 0 \\
h_{1,n} = Q_{food,i,n} - \left(1-r_{loss,food,storage,n-1}\right) Q_{food,i,n-1} - \sum_j q_{food,transp,j \rightarrow i,sell,n}^S \nonumber \\
+ \sum_j q_{food,transfer,i \rightarrow j,buy,n}^S - q_{food,store,i,n}^S + q_{food,market,i,n}^S = 0\\
h_{2,n} = Q_{food,max,i,n} - Q_{food,max,i,n-1} - \Delta Q_{food,storage,i,n} = 0
\end{gather}

\noindent with decision variables

\begin{equation}
q_{food,market,i}^S, q_{food,transp,j \rightarrow i,sell}^S, q_{food,transfer,i \rightarrow j,buy}^S, q_{food,store,i}^S \Delta Q_{food,storage,i}
\end{equation}

\noindent and state variables

\begin{equation}
Q_{food,max,i}, Q_{food,i}
\end{equation}

\noindent where

\begin{gather}
R = \sum_{food} \left(p_{food,market,i} q_{food,market,i}^S + \sum_j p_{food,transfer,i \rightarrow j,buy} q_{food,transfer,i \rightarrow j,buy}^S \right) \\
C = \sum_{food} \left( p_{food,store,i} q_{food,store,i}^S + \sum_j p_{food,transp,j \rightarrow i,sell} q_{food,transp,j \rightarrow i,sell}^S \right) \nonumber \\
+ \sum_{food} C_{food} Q_{food,i} + \sum_{food} p_{storage,food,expansion} \Delta Q_{food,storage,i} \\
r_{loss,food,storage} = 1 - \exp \left( \frac{- t_{month} r_{month} \ln 2}{\alpha_{q,storage} \tau_{food}}\right)
\end{gather}

The relevant pieces of derivative information are

\begin{gather}
\frac{\partial U}{\partial q_{food,market,i}^S} = p_{food,market,i} \\
\frac{\partial h_1}{\partial q_{food,market,i}^S} = 1 \\
\frac{\partial U}{\partial q_{food,transp,j \rightarrow i,sell}^S} = - p_{food,transp,j \rightarrow i,sell} \\
\frac{\partial h_1}{\partial q_{food,transp,j \rightarrow i,sell}^S} = -1 \\
\frac{\partial U}{\partial q_{food,transfer,i \rightarrow j,buy}^S} = p_{food,transfer,i \rightarrow j,buy} \\
\frac{\partial h_1}{\partial q_{food,transfer,i \rightarrow j,buy}^S} = 1 \\
\frac{\partial U}{\partial q_{food,store,i}^S} = - p_{food,store,i} \\
\frac{\partial h_1}{\partial q_{food,store,i}^S} = -1 \\
\frac{\partial U}{\partial \Delta Q_{food,storage,i}} = -p_{storage,food,expansion} \\
\frac{\partial g_2}{\partial \Delta Q_{food,storage,i}} = 1 \\
\frac{\partial h_2}{\partial \Delta Q_{food,storage,i}} = -1 \\
\frac{\partial g_1}{\partial Q_{food,max,i}} = -1 \\
\frac{\partial h_{2,n}}{\partial Q_{food,max,i,n}} = 1 \\
\frac{\partial h_{2,n}}{\partial Q_{food,max,i,n-1}} = -1 \\
\frac{\partial U}{\partial Q_{food,i}} = -C_{food} \\
\frac{\partial g_1}{\partial Q_{food,i}} = 1 \\
\frac{\partial g_3}{\partial Q_{food,i}} = -1 \\
\frac{\partial h_{1,n}}{\partial Q_{food,i,n}} = 1 \\
\frac{\partial h_{1,n}}{\partial Q_{food,i,n-1}} = - \left(1-r_{loss,food,storage}\right)
\end{gather}

The MCP equations to solve are

\begin{gather}
\frac{\partial U}{\partial q_{food,market,i}^S} - p_{food,store,eff} \frac{\partial h_1}{\partial q_{food,market,i}^S} = 0 \\
\frac{\partial U}{\partial q_{food,transp,j \rightarrow i,sell}^S} - p_{food,store,eff} \frac{\partial h_1}{\partial q_{food,transp,j \rightarrow i,sell}^S} = 0 \\
\frac{\partial U}{\partial q_{food,transfer,i \rightarrow j,buy}^S} - p_{food,store,eff} \frac{\partial h_1}{\partial q_{food,transfer,i \rightarrow j,buy}^S} = 0 \\
\frac{\partial U}{\partial q_{food,store,i}^S} - p_{food,store,eff} \frac{\partial h_1}{\partial q_{food,store,i}^S} = 0 \\
\frac{\partial U}{\partial \Delta Q_{food,storage,i}} - p_{storage,exp,lim} \frac{\partial g_2}{\partial \Delta Q_{food,storage,i}} - p_{storage,exp} \frac{\partial h_2}{\partial \Delta Q_{food,storage,i}} = 0 \\
- p_{storage,food,n} \frac{\partial g_{1,n}}{\partial Q_{food,max,i,n}} - p_{storage,exp,n} \frac{\partial h_{2,n}}{\partial Q_{food,max,i,n}} \nonumber \\
- p_{storage,exp,n+1} \frac{\partial h_{2,n+1}}{\partial Q_{food,max,i,n}} = 0 \\
\frac{\partial U}{\partial Q_{food,i}} - p_{storage,food,n} \frac{\partial g_{1,n}}{\partial Q_{food,i,n}} - p_{storage,empty,n} \frac{\partial g_{3,n}}{\partial Q_{food,i,n}} \nonumber \\
 - p_{storage,food,eff,n} \frac{\partial h_{1,n}}{\partial Q_{food,i,n}} - p_{storage,food,eff,n+1} \frac{\partial h_{1,n+1}}{\partial Q_{food,i,n}} = 0 \\
0 \leq -g_1 \perp p_{storage,food} \geq 0 \\
0 \leq -g_2 \perp p_{storage,exp,lim} \geq 0 \\
0 \leq -g_3 \perp p_{storage,empty} \geq 0 \\
q_{food,market,i}^S - \int N_{pop} q_{food,market,i}^A \left(I\right) \rho_I \left(I\right) dI = 0 \ , \ p_{food,market,i} \ \text{free} \\
h_1 = 0 \ , \ p_{storage,food,eff} \ \text{free} \\
h_2 = 0 \ , \ p_{storage,exp} \ \text{free}
\end{gather}

\subsection{Access/Consumer}

The optimization for consumers is

\begin{gather}
\max \sum_{t=1}^{12} U\left(t\right) + \sum_{t=13}^{24} \delta \tilde{U}\left(t\right) + \sum_{t=25}^{36} \delta^2 \hat{U}\left(t\right) \\
U \left(t\right) = \mathbf{q}_{food,t}^T \boldsymbol \gamma_{nutrient}^T \mathbf{B}^{-1} \mathbf{a} \left(I\right) - \frac{1}{2r_{month}} \mathbf{q}_{food,t}^T \boldsymbol \gamma_{nutrient}^T \mathbf{B}^{-1} \boldsymbol \gamma_{nutrient} \mathbf{q}_{food,t} \nonumber \\ 
- \mathbf{p}_{food,t}^T \mathbf{q}_{food,t} - \frac{1}{2} p_{change} \left\|\frac{\mathbf{q}_{food,t}}{r_{month,t}} - \frac{\mathbf{q}_{food,t-1}}{r_{month,t-1}}\right\|^2 \\
\end{gather}

\noindent with $q_{food}$ = $q_{food,market,i}^A$, $p_{food}$ = $p_{food,market,i}$; the notation is simplified for readability.

The relevant pieces of derivative information are

\begin{gather}
\frac{\partial U \left(t\right)}{\partial \mathbf{q}_{food,t}} = \boldsymbol \gamma_{nutrient}^T \mathbf{B}^{-1} \mathbf{a} \left(I\right) - \frac{1}{r_{month,t}} \boldsymbol \gamma_{nutrient}^T \mathbf{B}^{-1} \boldsymbol \gamma_{nutrient} \mathbf{q}_{food,t} - \mathbf{p}_{food,t} \nonumber \\
- p_{change} \left( \frac{1}{r_{month,t}^2} \mathbf{q}_{food,t} - \frac{1}{r_{month,t} r_{month,t-1}} \mathbf{q}_{food,t-1} \right) \\
\frac{\partial U \left(t+1\right)}{\partial \mathbf{q}_{food,t}} = - p_{change} \left( \frac{1}{r_{month,t}^2} \mathbf{q}_{food,t} - \frac{1}{r_{month,t} r_{month,t+1}} \mathbf{q}_{food,t+1} \right)
\end{gather}


The MCP equations to solve are

\begin{gather}
\frac{\partial U \left(t\right)}{\partial \mathbf{q}_{food,t}} + \frac{\partial U \left(t+1\right)}{\partial \mathbf{q}_{food,t}} = 0
\end{gather}

Dropping the descriptive subscripts and writing this in index notation (so as to make use of GAMS summation format)

\begin{gather}
\gamma_{lj} B^{lm} a_m - \frac{1}{r_{month,t}} \gamma_{lj} B^{lm} \gamma_{mk} q_k^{\left(t\right)} - p_j^{\left(t\right)} \nonumber \\
- p_{change} \left( \frac{1}{r_{month,t}^2} q_j^{\left(t\right)} - \frac{1}{r_{month,t} r_{month,t-1}} q_j^{\left(t-1\right)}\right) \nonumber \\
- p_{change} \left( \frac{1}{r_{month,t}^2} q_j^{\left(t\right)} - \frac{1}{r_{month,t} r_{month,t+1}} q_j^{\left(t+1\right)} \right) = 0
\end{gather}

\section{Utilization}

There is no optimization for utilization.

\begin{gather}
q_{food,a,g,I} \left(a,g,I\right) = \left(1-r_{loss,food,utilization}\right) \frac{w\left(a,g\right)}{\mu_{w|I}} q_{food,market,i}^A \left(I\right) \\
\mu_{w|I} = \int w\left(a,g\right) \rho_{a,g} \left(a,g|I\right) da dg \\
\boldsymbol \nu_{actual} = \boldsymbol \nu \left( q_{food,a,g,I} \right) \\
\boldsymbol \nu_{working} = \int \boldsymbol \nu_{actual} \left(a \geq 16, g = \text{male},I\right) \rho_I \left(I\right) dI \\
r_{labour} = 2 - \boldsymbol \alpha_{labour}^T \boldsymbol \nu_{working}
\end{gather}


\section{Model Simplifications}

In order to simplify the model (for ease of calibration and to reduce run-time), we made several changes.

\begin{enumerate}
\item The expansion variables, constraints, and prices (e.g. to increase total crop area, transportation capacity, etc.) were removed.
\item Cropping and dietary conservatism were removed -- penalties for changing yearly crop areas and monthly consumption, respectively.
\item The equations disaggregating consumption by age and gender, along with $r_{labour,ave}$ were removed.
\item The model now solves in yearly time slabs rather than all at once.
\item Food transportation now only takes place between storage facilities in different nodes -- not from production in one node to storage in another.  This effectiely gets rid of $q_{food,transp,buy,D}$
\end{enumerate}

We also made some changes to improve model convergence:

\begin{itemize}
\item multiplied the KKT for $A_{crop}$ by -1
\item turned $h_{livestock,1}$ into a $\geq$ rather than an $=$
\item turned the KKTs for $q_{food,transfer,buy,D}$ and $q_{cattle,transp,buy,D}$ into a $\geq$ rather than an $=$
\item turned $h_{storage,1}$ into a $\geq$ rather than an $=$
\item turned the KKT for $q_{food,market,A}$ into a $\geq$ rather than an $=$
\item rearranged the KKT for $q_{food,market,A}$ to use $B_f$ rather than $B_f^{-1}$
\end{itemize}

\section{Model Decomposition}

\subsection{Model Partitioning and Coordination}

Because the full ten-node model was too big to solve on its own, we had to come up with a way to decompose it.  The problem as a whole can be thought of as

\begin{gather}
0 \leq F \left(q, p, p_s\right) \perp \left\{q,p,p_s\right\} \geq 0 \\
q = \left\{ \begin{array}{c}
q_0 \\
q_1 \\
\vdots \\
q_n \end{array} \right\}, p = \left\{ \begin{array}{c}
p_0 \\
p_1 \\
\vdots \\
p_n \end{array} \right\}, p_s = \left\{ \begin{array}{c}
p_{1s} \\
p_{2s} \\
\vdots \\
p_{ns} \end{array} \right\}
\end{gather}

\noindent where $q_i$ are the flows/primal variables, $p_i$ are the prices/dual variables from the optimizations, and $p_s$ are the market clearing prices.

The combined production, storage, and consumption subproblem for node $i$ is

\begin{equation}
0 \leq F_i \left(q_i, p_i, p_{is}\right) \perp \left\{q_i, p_i\right\} \geq 0
\end{equation}

\noindent where the distribution problem is

\begin{equation}
0 \leq F_0 \left(q_0, p_0, p_s\right) \perp \left\{q_0, p_0\right\} \geq 0
\end{equation}

\noindent and where the market clearing (with distribution) subproblems for each node $i$ is 

\begin{equation}
0 \leq F_{is} \left(q_i, q_0 \right) \perp \left\{p_{is}\right\} \geq 0
\end{equation}

Then decompose the problem into node subproblems and a distribution master problem.  In parallel, solve the production/storage/consumption problem and the market clearing problem with $q_0$ fixed; this is the node-level subproblem.  The distribution problem is further broken down, modified to incorporate a coordination scheme, and then solved with $p_s$ fixed.

\begin{equation}
0 \leq F_0 \left(q_0, p_0, p_s\right) \perp \left\{q_0, p_0\right\} \geq 0 \rightarrow \left\{\begin{array}{c}
0 \leq F_{0,eq} \left(q_0, p_0, p_s\right) = 0 , q_0 \ \text{free} \\
0 \leq F_{0,ineq} \left(q_0, p_0, p_s\right) \perp p_0 \geq 0
\end{array} \right.
\end{equation}

In the case of this model, $F_{0,eq}$ simplifies down to $F_{0,eq} \left(p_0, p_s\right)$

Now modify $F_{0,eq}$:

\begin{equation}
0 \leq F_{0,eq} \left(p_0, p_s\right) + W \left(q_0^{\left(k+1\right)} - q_0^{\left(k+1\right)}\right) = 0
\end{equation}

\noindent where $W$ is a weighting matrix.  If $q_0^{\left(k+1\right)} \rightarrow q_0^*$, then 

\begin{align}
W \left(q_0^{\left(k+1\right)} - q_0^{\left(k+1\right)}\right) \rightarrow 0 \\
F_{0,eq} \left(p_0, p_s\right) \rightarrow 0
\end{align} 

\subsection{Model Simplification}

This decomposition means that we can pass some variables as parameters (i.e. as constants).  Market clearing prices are passed as parameters to the distribution problem, and distribution quantities are passed as parameters to the nodal subproblems.  In addition to that, $C_{labour}$ can also be passed as a parameter to the distribution problem.  This means that for the nodal subproblems, some of the equations and variables can be removed by substitution into simple linear equations (often $x-y_{const}=0$ type equations); this reduces the size of the subproblem and thus speeds up convergence.  Furthermore, this allows us to construct an indicator parameter for distribution.  For cattle, this parameter is

\begin{equation}
p_{ind,cattle,i \rightarrow j} = p_{cattle,transp,i \rightarrow j,sell} - p_{cattle,transp,i \rightarrow j,buy} - m_{cow} \left(C_{fuel} + C_{labour}\right)
\end{equation}

\noindent whereas for food, it is

\begin{equation}
p_{ind,food,i \rightarrow j} = p_{food,transp,i \rightarrow j,sell} - p_{food,transp,i \rightarrow j,buy} - \left(C_{fuel} + C_{labour}\right)
\end{equation}

If $p_{ind,food,i \rightarrow j}$, then the distributor will want to ship food from $i$ to $j$; the same holds for $p_{ind,cattle,i \rightarrow j}$ and cattle.  The coordination part of the distribution problem then becomes

\begin{equation}
- p_{ind} + p_{transp} + W \left(q_0^{\left(k+1\right)} - q_0^{\left(k+1\right)}\right) \geq 0
\end{equation}

\subsection{Weighting Choices and Convergence}

Consider the simple case where there is no limit on transportation quantities (i.e. $p_{transp} = 0$).  Then

\begin{equation}
q^{\left(k+1\right)} = W^{-1} p_{ind} + q^{\left(k\right)}
\end{equation}

\noindent with the limitation that $q^{\left(k\right)} \geq 0$; trade only flows where $p_{ind} > 0$.  This iterative process is guaranteed to converge to $q^*$ if the process starts close enough and if 

\begin{equation}
\left| \text{eig} \left(I + \frac{\partial}{\partial q} \left(W^{-1} p_{ind}\right) \right) \right| < 1
\end{equation}

\noindent at $q = q^*$.  If $W$ is a constant diagonal matrix, then the convergence criterion simplifies down to

\begin{equation}
\left| 1 + \frac{1}{w} \frac{\partial p_{ind}}{\partial q} \right| < 1
\end{equation}

Since $p_{ind}$ is a measure of the price disparity between regions, and since shipping commodities from a low price region to a high price region will raise prices in the first region and lower them in the second region, we can be confident that $\frac{\partial p_{ind}}{\partial q} < 0$.  Therefore, if we choose an appropriate weight $w$, we will converge; too small a $w$ value will make the iteration unstable, and too large a $w$ value will make the iteration progress very slowly.

To make the units compatible and to avoid problems associated with differences of scale, we have used

\begin{align}
w_{food} \quad &= \quad  \alpha_{food} n_{nodes} n_{food} \frac{p_{buy} + p_{sell}}{2} \frac{1}{\max \left(q_{food},1 \right)} \\
w_{cattle} \quad &= \quad  \alpha_{cattle} n_{nodes} \frac{p_{buy} + p_{sell}}{2} \frac{1}{\max \left(q_{cattle},1 \right)}
\end{align}

\noindent where the $\alpha$ parameters are adjustable, $n_{food}$ is the number of food commodities, $n_{nodes}$ is the number of nodes used in the model, the prices are of the commodity and regions in question, and the quantities are the amount being transported; the formula is adjusted to avoid any divide-by-zero errors.  Using prices like this scales the price discrepancy with respect to the magnitudes of the prices involved, and using the quantity transported like this scales the potential change in that quantity so that it is relative to how much is already being transported.  The weight's magnitude also increases with the number of nodes and food commodities to maintain stability as the model size increases.

This convergence analysis is, of course, a simplification.  Firstly, we ignore any cross sensitivities between commodities (e.g. shipping wheat affects the price of peppers); the assumption is that any such sensitivities are relatively small.  Secondly, we have ignored the transportation capacity constraints and $p_{transp}$.  If those capacity constraints are reached, $p_{transp}$ should converge to $p_{ind}$ as the commodity quantities stabilize at their ceilings:

\begin{equation}
W \left(q_0^{\left(k+1\right)} - q_0^{\left(k+1\right)}\right) \rightarrow 0 \Rightarrow p_{transp} \rightarrow p_{ind}
\end{equation}

Proving this convergence in a manner similar to that done above for the case without transportation capacity constraints, however, would be much more complicated.

\section{Parameter Calibration}

\subsection{Calibration Methods}

In the 'Variables and Parameters' spreadsheet, the parameters in the 'Parameters to be Calibrated' sheet have had their values determined either by a WAG or by some form of calibration.  A WAG is an estimate of what a reasonable value might be.  The calibrated values have been determined by an internal or external calibration.  

An external calibration, under this scheme, means that the values have simply been lifted from an outside source and put into the model.  For example, the region populations and areas were obtained from Wikipedia and then used in the model.  As another, more complicated example, the parameters related to dairy production (\verb!r_dairy!, \verb!mu_production!) were determined using data from a relevant paper (bohmanova07jsr).  The key point is that these parameters were determined directly and/or without running the full MCP model.

An internal calibration means that the values for certain parameters have been determined by matching model outputs to known data; the values have been iteratively adjusted to produce certain model results.  For example, the \verb!a_tilde! values were chosen so that the nutrient intake and diet composition results produced by the model were similar to the food consumption survey reports.


\subsection{Calibration Insights}

Stabilizing cattle populations is very difficult -- either the population drops off significantly at first and then levels off, or you get spiking prices and no one eats beef.  Keeping consumption down to the right levels is also difficult.  It's probably easiest to tweak $\tilde{a}_{food}$ and $B_f$, but keep in mind that the cattle population can be strongly influence by demand for milk consumption.

If a region has $m_{crop} = 0$ for all of its crops in a given year, the model will likely fail to solve.  It can be hard to get the proper spread of $m_{crop}$ values, though -- it's easy to get values that are too high or too low/equal to 0.

\section{Intervention Modelling}

There are (at least) five different types of interventions that can be investigated with the model.  The first four have costs associated with them which can be directly calculated in post-processing for cost-benefit analysis; the last would require further research to associate a cost with the intervention.

\begin{itemize}
\item Direct Food Aid.  This directly deposits food into storage in a region ($q_{food,aid}$).  The changed equation is
\begin{align}
Q_{food,i,n} - \left(1-r_{loss,food,storage,n-1}\right) Q_{food,i,n-1} - \sum_j q_{food,transp,j \rightarrow i,sell,n}^S \nonumber \\
+ \sum_j q_{food,transfer,i \rightarrow j,buy,n}^S - q_{food,store,i,n}^S + q_{food,market,i,n}^S - q_{food,aid} = 0
\end{align}
\item Direct Cash Aid.  This gives people money to temporarily increase their income ($I_{cash,transfer}$).  The demand curve intercept for a given month in which direct cash aid is given is then
\begin{equation}
\mathbf{a} = \log \left(I + n_{months} I_{cash,transfer}\right) \tilde{\mathbf{a}}
\end{equation}
\item Consumer Subsidy.  The government pays a percentage of the food price for the consumer ($r_{sub,consumer}$).  The changed equation is
\begin{equation}
- a_{food} + q_{food,market} + B_f \left( 1 - r_{sub,consumer} \right) p_{food,market} \geq 0
\end{equation}
\item Producer Subsidy.  The government pays an extra percentage to farmers for food produced ($r_{sub,producer}$).  The changed equation for a given food (crops, beef, or milk) is
\begin{equation}
p_{food,store} \left( 1 + r_{sub,producer}\right) - p_{food} = 0
\end{equation}
\item Transportation Expansion.  Transportation capacity between (some) nodes is increased ($q_{transp,exp}$).  The changed equation is
\begin{gather}
\sum_{food} \left( q_{food,transp,i \rightarrow j,buy}^D + q_{food,transfer,i \rightarrow j,buy}^D \right) + m_{cow} q_{cattle,transp,i \rightarrow j,buy}^D \nonumber \\ 
- \left(q_{transp,capacity} + q_{transp,exp} \right) \leq 0
\end{gather}

\end{itemize}

\section{Code Documentation}

Code related to the model is written up in both GAMS and MATLAB.  The GAMS code solves the MCP while the MATLAB code calculates some parameters for use in the MCP and processes the MCP results (e.g. for producing graphics) afterwards.

There are several relevant MATLAB m-files

\begin{itemize}
\item \verb!par_calc! calculates parameters used in the MCP and for post-processing
\item \verb!par_write! takes the parameters from \verb!par_calc! and writes them to gdx files
\item \verb!data_read! and \verb!data_read_dist! read the MCP outputs from the gdx files produced by the non-decomposed and decomposed models, respectively
\item \verb!data_plot! uses the MCP outputs read by \verb!data_read!/\verb!data_read_dist! and the calculated parameters to produce figures and graphs
\item \verb!trade_flows! uses the MCP outputs read by \verb!data_read!/\verb!data_read_dist! to produce diagrams of the trade flows between different regions
\end{itemize}

The relevant GAMS files are associated with either the `small' or the `distributed' Ethiopia MCP problems.

The `small' problem uses no decomposition but solves the MCP in time slabs.  The main file is \verb!MCP Food Model small!.  Within that file, there are \verb!#include! statements for declaring sets, parameters, variables, equations, and the MCP definition:

\begin{itemize}
\item \verb!Set Declarations small! defines the sets involved in the MCP.  For testing different scenarios, the most important sets are \verb!n_active! and \verb!y_active!; these are the sets of nodes and years that will be solved over.
\item \verb!Parameter Declarations small! declares and defines parameter values for all parameters that are indexed by, at most, one set.  These are grouped by category (e.g. demography, livestock, etc.).
\item \verb!Table Declarations small! declares and defines all parameters that are indexed by two sets.
\item \verb!Parameter Calculations small! does some simple calculations to produce new parameters or to modify existing ones.
\item \verb!Scenario Parameters small! imports parameters from MATLAB and defines intervention parameters for a given scenario.
\item \verb!Variable Declarations small! declares the variables used in the model.  These are divided up into free and positive variables depending on whether they match up with equalities or inequalities, respectively.  These are grouped by category (e.g. demography, livestock, etc.).
\item \verb!Equation Calculations small! declares and defines the equations to be solved in the MCP.  These are grouped by category (e.g. demography, livestock, etc.).
\item \verb!MCP Definition Small! matches up the variables with their respective equations to define the whole MCP.  Note that for state variables, variable \verb!variable_name! is matched with \verb!variable_name_eqn!, for decision variables with KKT conditions, \verb!variable_name! is matched with \verb!L_variable_name!, and for constraints, each constraint is matched with its shadow price.
\end{itemize}

Note that within \verb!MCP Food Model small!, there are several parameters declared which as used as yearly versions of the parameters. 

The `distributed' problem decomposes the MCP into nodal subproblems and a distribution master problem while still solving in time slabs.  The main file is `MCP Food Model distributed - parallel'; it solves the nodal subproblems in parallel using the \verb!model handles! syntax and associated code.  Within each year, the code iterates between the nodal and distribution problems until convergence.  The nodes are solved first, the coordination weights and indicators are calculated, and then the distribution problem is solved.  At each such step, the convergence measures and indicators are recorded, but the usual GAMS output is mostly eliminated to avoid having an output file which is gigabytes of text.  The different convergence measures measure different things:

\begin{itemize}
\item \verb!conv! measures the $L_2$ norm of the non-converged indicators
\item \verb!conv2! measures the $L_{\infty}$ norm of the non-converged indicators
\item \verb!conv3! measures the $L_2$ norm of the change in distribution quantities
\end{itemize}

The set declarations, etc. are similar to those in the `small' model with the exception that there are separate equation declarations and MCP definitions for the nodal and distribution problems.

\section{Model Foresight Modifications}

Right now, the model assumes perfect foresight (within the three year rolling time horizon) regarding things like crop yield.  When there is, e.g., a crop failure, even if the model only `sees' that failure in the year that it happens (i.e. not in a previous advisory year), the model will still try to make significant adjustments to mitigate the shock.  These adjustments include the crop producers increasing the crop areas significantly to try and produce enough food while the storage operators immediately start holding back food at the beginning of the year and storing food across years to make the food last; this causes a price jump at the beginning of the year -- well before the bad harvest actually happens.  Therefore, the model has been altered in two small ways.

Firstly, the total crop area available has been significantly reduced so that an average yield has the total crop area used at a value near that ceiling; that way, there isn't room to significantly increase crop area (though the relative areas of different crops may change).  In the model, the crop producers still knows what's going to happen, but the cropland limitation restricts their ability to do anything about it.  Secondly, there is now a constraint which forces storage to store no more than a certain quantity of food right before harvest:

\begin{equation}
\sum_{food} Q_{food} \leq Q_{food,clear}
\end{equation}

This constraint only applies to the month before the expected harvest, $m_{clear}$.  In a region with two crops, the cropping month under consideration is the latest crop in the year (usually the kremt harvest, but if there is no kremt harvest, it would be the month of the belg harvest).  This constraint has a shadow price $p_{clear}$  which shows up in the KKT condition for $Q_{food}$ associated with the month $m_{clear}$ in each region.  Analogously to the previous modification, the storage operator knows the future but is constrained so that that knowledge cannot be fully utilized. 

% BibTeX users please use one of
%\bibliographystyle{spbasic}      % basic style, author-year citations
%\bibliographystyle{spmpsci}      % mathematics and physical sciences
%\bibliographystyle{spphys}       % APS-like style for physics
%\bibliography{}   % name your BibTeX data base

\end{document}
